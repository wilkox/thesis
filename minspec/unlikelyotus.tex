\begin{table}
\small
\caption[Examples of spurious species identifications]{Selected examples of species identified in a marine metagenome using the \naive{} method.
These species were identified in a single sample from the \ac{SO} (sample 346; see \ref{chp:polarfront}).
The sample was compared to the RefSeq database of full genomes using \softwarename{tblastx} with an E-value maximum of 1.0 $\times$ 10\textsuperscript{$-$3}, i.e.\ only high-quality hits were included.
Relative abundances were calculated using \ac{GAAS} \cite{Angly:2009ip}.
}
\label{tab:unlikelyotus}
\smallskip
\begin{center}
\begin{tabularx}{\textwidth}{lll}
\toprule
\textbf{Species} & \textbf{Relative} & \textbf{Notes}\\
& \textbf{Abundance (\%)}&\\
\midrule
Encephalomyocarditis virus & 1.98 & Human pathogen.\\
Marek's disease virus type 1 & 1.49 & Chicken pathogen.\\
Marek's disease virus type 2	& 0.85 & Chicken pathogen.\\
\speciesfull{Francisella philomiragia}& 0.041 & Human and animal pathogen.\\
\speciesfull{Agrobacterium vitis} & 0.040 & Plant and opportunistic human pathogen.\\
\speciesfull{Brucella suis} & 0.011 & Human and swine pathogen (causes brucellosis).\\
\genus{Enterobacter} sp. 638	& 0.0085 & Animal commensal/pathogen.\\
\speciesfull{Bordetella parapertussis} & 0.0075 & Mammalian pathogen (causes mild form of whooping cough).\\
\speciesfull{Neisseria meningitidis} & 0.0074 & Human pathogen.\\
\speciesfull{Yersinia pestis} & 0.0060 & Human/animal pathogen (causes bubonic plague).\\
\bottomrule
\end{tabularx}
\end{center}
\end{table}
