\chapter{General discussion}
\label{ch:generaldiscussion}

\section{Summary}

This thesis aimed to investigate the microbial ecology and biogeography of the \ac{SO}, with two main areas of focus.
To achive this, two factors that structure the biogeographic distribution of microorganisms in the \ac{SO} were selected for study: the \ac{PF}, a major biogeographic barrier, and advection, a potentially major biogeographic force.

\subsection{The Polar Front}

This project found good evidence that the \ac{PF} is a major biogeographic barrier in the \ac{SO}, by demonstrating that the bacterial and archaeal communities in the waters to the south (the \ac{AZ}) are significantly different from the waters to the north (the \ac{SAZ} and subtropical waters north of the \ac{STZ}, primarily representing \ac{SAMW}) (\secreft{ch:polarfront}).
This is not the first study of the effect of the \ac{PF} on the distribution of \ac{SO} microbiota.
Variation in the position of the \ac{ACC}, which determines the location of the \ac{PF}, has been shown to influence zooplankton composition \citep[e.g.][]{Chiba:2001un,Hunt:2001vp}, including dinoflagellates \cite{Esper:2002ui}\footnote{Interestingly, one study desribed the biogeographic effect of \ac{ACC}-associated fronts on zooplankton as being only partially explicable by environmental parameters, suggesting the effect of advection (\secreft{ch:advection}) on zooplankton as a potential avenue for future research}.
The \ac{ACC} and/or \ac{PF} have similarly been shown to influence the distribution of Roseobacter phylotypes \cite{Selje:2004ka,Giebel:2009hr}, Flavobacteria \cite{Abell:2005ji}, and SAR11 phylotypes \cite{Giebel:2009hr}, and have been predicted to partition phytoplankton distribution as a result of the advective distribution of nutrients \cite{Weber:2010fi}.
However, this thesis desribes the first community-level (metagenomic) survey of \ac{SO} bacterial and archaeal plankton performed over a latitudinal transect occupying all major \ac{SO} surface water masses to give an integrated ``snapshot'' of the microbial ecology and biology of the \ac{SO}.

As well as confirming these patterns, this study found that the effect of the \ac{PF} extends to the whole-community level, and even to the distribution of genomically encoded functional potential.
The higher abundance south of the \ac{PF} of Bacteroidetes and Rhodobacterales, associated with the degradation of phytoplanktonic byproducts \citep[e.g.][]{Buchan:2005hd,Williams:2012gsa}, reflect the higher concentrations of (primarialy eukaryotic) phytoplankton in this region. 
This was also reflected in the functional analysis, with an overrepresentation of high-specifity transporters, suggestive of copiotropic taxa in a ``feast'' phase \cite{Lauro:2009gx}.
In general, waters south of the \ac{PF} reflected the upwelling of nutrient-rich \ac{NADW} and higher supply of light during the austral summer, which make the region significantly more active and productive than \ac{SAZ} and even subtropical waters to the north.

These northern waters were characterised by a higher relative abundance of slow-growing, nutrient-scavenging oligotrophs such as SAR11 and SAR116.
Functionally, this was reflected in the higher abundance of genes encoding branched-chain amino acid transporters, which both SAR11 and SAR116 possess.
The other significant feature of region north of the \ac{PF} was the higher abundance of the cyanobacterial genera \genus{Prochlorococcus} and \genus{Synechococcus}, and concurrently the photosynthesis functions they encode.
This was most likely due to the sensitivity of these genera to temperature.

\subsubsection{Biogeographic role of the Polar Front}

Having shown that the \ac{PF} is a major biogeographic barrier, and in light of the advection study also presented in this thesis (\secreft{ch:advection}), it is worth considering the mechanism(s) by which the \ac{PF} shapes microbial biogeography.
It is likely that three main forces are at work.

The first is the role of the \ac{PF} as a biogeographic barrier in the classic sense in which it is usually applied to macroorganisms.
In other words, it physically prevents or slows the migration of cells between the regions it divides, leading if not to allopatric speciation, at least to some degree of genomic divergence, which is amplified by the differences in environmental properties between the two regions.
\citet{Selje:2004ka}, who first reported that \ac{RCA} phylotypes differed across the \ac{PF}, offered this mechanism as a likely explanation.
The advection effect (\secreft{ch:advection}) supports such a mechanism, by showing that oceanic regions poorly connected by advection (i.e.\ poorly mixed) have on average less similiar microbial communities.


Is it a barrier?
How do the communities differ compositionally?
How do the communities differ functionally?
Advection
Is there an advection effect?

Levels of description in microbial ecology and biogeography

"Top down" vs "bottom up": medicine started "top down"; economics started "bottom up"; microbial ecology would like to have started "top down", but methodologically was forced to proceed "bottom up".

'omics finally allows microbial ecology to become a "top down" discipline. But as this perspective is almost brand new, we're struggling with how to meaningfully carve up and interpret what we see. The species concepts seems natural and easily measured on the cells-and-petri-dishes level, but seems to become almost meaningless when examining a whole community. Moreover, the tiny size, incredible diversity and adaptiveness of microbes means enormous amounts of data must be collected and carefully processed to identify even the simplest of biogeographic patterns (e.g. distance effect).

climate change needs a mention here somewhere

Nevertheless, progress is being made: environment effect, distance effect, and now advection effect. The challenge is linking the top to the bottom. It's very hard to disentangle the effects of individual OTUs — give as examples surface vs. DCM; BVSTEP of advection results. Just as it might be misguided for a physician to try to conceptualise a disease in terms of the elemental ratios of diseased vs healthy tissue, it may not even be a fruitful approach to try and think of ecological patterns in terms of their constituent species.

Original contributions of this thesis

This thesis presents novel work which contributes to bridging the gap between the cellular and ecological levels. (minspec, polar front, advection effect). The methods and software which have been developed are also of general application.

Future work

SO microbial ecology
Current metagenomic methods require only looking at a small part of the community at a time (constrained by primer selection, size fractionation) - would be great to see the whole thing - single cell?


Future studies should seek to confirm the advection results, particularly the advection effect, in other regions.
