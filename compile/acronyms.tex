%define a new command, adef, for easy acronym definition
\newcommand{\adef}[2]{\newacronym{#1}{#1}{#2}}

%redefine acronym calls for ease
\newcommand{\ac}[1]{\gls{#1}}
\newcommand{\acp}[1]{\glspl{#1}}

%the actual list of acronyms
\adef{BW}{Bottom Water}
\adef{AAIW}{Antarctic Intermediate Water}
\adef{SB}{Southern Boundary of the Antarctic Circumpolar Current}
\adef{AABW}{Antarctic Bottom Water}
\adef{RCA}{Roseobacter Clade Affiliated}
\adef{THC}{Thermohaline Circulation}
\adef{IPY}{International Polar Year}
\adef{GOS}{Global Ocean Sampling}
\adef{SACCF}{Southern Antarctic Circumpolar Current Front}
\adef{AC}{Antarctic Convergence}
\adef{CDW}{Circumpolar Deep Water}
\adef{NADW}{North Atlantic Deep Water}
\adef{SSU}{Small Subunit}
\adef{FISH}{Fluorescence \emph{In Situ} Hybridization}
\adef{LWM}{Low Molecular Weight}
\adef{AP}{Antarctic Peninsula}
\adef{AOB}{Ammonia-Oxidizing Bacteria}
\adef{OMG}{Oligotrophic Marine Gammaproteobacteria}
\adef{KEOPS}{Kerguelen Ocean and Plateau Compared Study}
\adef{DGGE}{Denaturing Gradient Gel Electrophoresis}
\adef{SAF}{Subantarctic Front}
\adef{MMPA}{methylmercaptopropionate}
\adef{DFAA}{Dissolved Free Amino Acids}
\adef{MGI}{Marine Group I Crenarchaeota}
\adef{AAP}{Aerobic Anoxygenic Phototrophic}
\adef{DOM}{Dissolved Organic Matter}
\adef{POM}{Particulate Organic Matter}
\adef{HMW}{High Molecular Weight}
\adef{ACC}{Antarctic Circumpolar Current}
\adef{DOC}{Dissolved Organic Carbon}
\adef{DMSP}{dimethylsulfoniopropionate}
\adef{HNLC}{High Nutrient, Low Chlorophyll}
\adef{SO}{Southern Ocean}
\adef{STF}{Subtropical Front}
\adef{SAZ}{Subantarctic Zone}
\adef{AOA}{Ammonia Oxidizing Archaea}
\adef{ORF}{Open Reading Frame}
\adef{nMDS}{non-metric MultiDimensional Scaling}
\adef{CTD}{Conductivity, Temperature and Depth}
\adef{PFZ}{Polar Frontal Zone}
\adef{PF}{Polar Front}
\adef{CEAMARC}{Collaborative East Antarctic Marine Census}
\adef{CASO}{Climate of Antarctica and the Southern Ocean}
\adef{NZ}{North Zone}
\adef{SZ}{South Zone}
\adef{DZ}{Deep Zone}
\adef{AZ}{Antarctic Zone}
\adef{OTU}{Operational Taxonomic Unit}
\adef{ANOSIM}{Analysis of Similarities}
\adef{SIMPER}{SimMilarity Percentages}
\adef{KEGG}{Kyoto Encyclopedia of Genes and Genomes}
\adef{CFB}{Cytophaga-Flavobacterium-Bacteroides}
