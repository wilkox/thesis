\chapter{The Polar Front as a major biogeographic boundary in the Southern Ocean} 
\label{ch:polarfront}

Sections of this chapter have been previously published in \bibentry{Wilkins:2012td}.

\section{Summary}

\section{Introduction}


\section{Methods}
\subsection{Sampling and metagenomic sequencing}

Sampling\footnote{Sampling was performed by Jeffrey M. Hoffman and Jeffrey B. McQuaid} was conducted on board the RSV \emph{Aurora Australis} during cruise V3 \ac{CEAMARC/CASO} from 13 December 2007 -- 26 January 2008. 
This cruise occupied the SR3 latitudinal transect from Hobart, Australia (44\textdegree{} S) to the Mertz Glacier, Antarctica (67\textdegree{} S) within a longitudinal range of 140--150\textdegree{} E.
Nineteen samples (16 surface, 3 deep) were obtained along almost the entire latitudinal range \figref{fig:samplemap}.

% the sample map
\begin{figure}
  \centering
  \includegraphics[width=\textwidth]{../polarfront/samplemap.png}
  \caption{TODO caption here}
  \label{fig:samplemap}
\end{figure}


A range of data were recorded by integrated instruments on the RSV \emph{Aurora Australis} including location, water column depth, water temperature, salinity, fluorescence and meterological data \tabref{tab:samplelist}.
These data were used to locate the \ac{PFZ} based on a surface temperature gradient of \textapprox{} 1.35 \textdegree{}C across a distance of 45--65 km, placing the \ac{PF} at approximately $-59.70$\textdegree{} of latitude, consistant with previous descriptions \cite{Moore:1999to,Sokolov:2002tc}.
Samples were accordingly grouped into ``North'' and ``South'' zones, while the three deep samples composed a ``Deep'' zone \tabref{tab:samplelist}.
The \ac{NZ} represents waters from the Subtropical, Subantarctic and \ac{PFZ} regions, while the \ac{SZ} represents the \ac{AZ}.

\begin{landscape}
\begin{table}
\sffamily
\caption[Details of samples used in Polar Front study]{\sffamily{}Sampling time, location and physicochemical properties of samples used in this study.
All data were retrieved from underway instruments aboard the RSV \textit{Aurora Australis}.}
\label{tab:samplelist}
\begin{tabu} to\linewidth{llllXIXZXXX}
\toprule
\textbf{Sample} & \textbf{Zone} & \textbf{Date} & \textbf{Latitude} & \textbf{Longitude} & \textbf{Water Column Depth (m)} & \textbf{Sample Depth (m)} & \textbf{Temperature (\textdegree{}C)} & \textbf{Salinity (PSU)} & \textbf{Fluorescence (\textmu{}gL\textsuperscript{\textminus{}1})} & \textbf{Volume filtered (L)}\\
\midrule

346 & North & 20/12/2007 & \textminus{}59.31 & 142.59 & 4294 & 2 & 2.9 & 33.75 & 0.3 & 500\\
347 & South & 23/12/2007 & \textminus{}66.02 & 142.74 & 450 & 2 & 0.6 & 34.20 & 4.0 & 250\\
349 & South & 27/12/2007 & \textminus{}66.57 & 142.32 & 370 & 1.5 & \textminus{}1.3 & 34.40 & 2.3 & 250\\
351 & South & 28/12/2007 & \textminus{}66.56 & 143.43 & 823 & 1.5 & \textminus{}0.6 & 34.30 & 1.3 & 500\\
352 & South & 29/12/2007 & \textminus{}66.77 & 143.32 & 164 & 2.5 & \textminus{}0.8 & 34.30 & 3.1 & 500\\
353 & South & 30/12/2007 & \textminus{}67.05 & 144.68 & 180 & 2 & \textminus{}1.8 & 34.40 & 0.3 & 500\\
357 & South & 05/01/2008 & \textminus{}66.17 & 143.02 & 580 & 2 & \textminus{}0.4 & 34.15 & 2.5 & 500\\
358 & South & 09/01/2008 & \textminus{}64.30 & 150.03 & 3550 & 2 & 0 & 33.55 & 0.5 & 500\\
359 & South & 12/01/2008 & \textminus{}66.19 & 143.53 & 540 & 2 & \textminus{}0.2 & 34.21 & 2.5 & 500\\
360 & South & 13/01/2008 & \textminus{}66.58 & 141.02 & 316 & 2 & \textminus{}0.7 & 34.04 & 6.2 & 500\\
362 & South & 19/01/2008 & \textminus{}65.54 & 140.83 & 1064 & 2 & 0.7 & 32.20 & 0.5 & 500\\
363 & North & 22/01/2008 & \textminus{}60.00 & 141.31 & 4473 & 2 & 3.3 & 33.77 & 0.1 & 500\\
364 & North & 23/01/2008 & \textminus{}56.70 & 141.88 & 3693 & 2 & 4 & 33.70 & 0.5 & 500\\
366 & North & 24/01/2008 & \textminus{}52.02 & 144.14 & 3180 & 2 & 7.6 & 33.84 & 0.3 & 500\\
367 & North & 25/01/2008 & \textminus{}48.25 & 145.90 & 3490 & 2 & 11 & 34.43 & 0.2 & 500\\
368 & North & 26/01/2008 & \textminus{}44.72 & 145.78 & 3201 & 2 & 14.8 & 34.96 & 1.3 & 560\\

\bottomrule
\end{tabu}
\end{table}
\end{landscape}


At each station, \textapprox{} 250--560 L of seawater was pumped from \textapprox{} 1.5--2.5 m below the sea surface into drums stored at ambient temperature on deck. 
In the case of deep samples, \textapprox{} 225--230 L of seawater was collected from Niskin bottles attached to a \ac{CTD} (SeaBird, Bellevue, USA).
Seawater samples were prefiltered through a 20 \micron{} plankton net, then filtrate was captured on sequential 3.0 \micron{} 0.8 \micron{} and 0.1 \micron{} 293 mm polyethersulfone membrane filters (Port Washington, USA), and immediately stored at $-20$ $^\circ$C \cite{Rusch:2007ez,Ng:2010cd}.

DNA extraction\footnote{DNA extraction was performed by Cynthia Andrews-Pfannkoch and others at the J. Craig Venter Institute} was performed at the J. Craig Venter Institute (Rockville, USA) as described in \citet{Rusch:2007ez}.
Pyrosequencing was performed on a GS20 FLX Titanium instrument (Roche, Branford, USA) also at the J. Craig Venter Institute as described in \citet{Lauro:2010jna}.
Duplicate reads and reads with many pyrosequencing errors were removed as described in \citet{Lauro:2010jna}.

\subsection{Phylogenetic analysis of metagenomic data}

\subsubsection{\softwarename{blast} comparison to RefSeq database}

A subset of the RefSeq microbial (bacterial and archaeal) genome database (release 41, retrieved May 31 2012 from \url{ftp://ftp.ncbi.nih.gov/refseq/release/}) was prepared by excluding sequences with the words ``shotgun'', ``contig'', ``partial'', ``end'' or ``part'' in their headers \cite{Angly:2009ip}.
Because this database was not expected to contain representative genomes for every species present, \acp{OTU} in this study are defined by the best species match to this database, and may for example represent congeners.

The metagenomic reads from each sample were compared against this database using \softwarename{tblastx}, with default parameters except for: E-value threshold $1.0\times{}10^{-3}$, cost to open gap 11, cost to extend gap 1, masking of query sequence by \softwarename{SEG} masking with lookup table only.
The outputs of all \softwarename{tblastx} searches against RefSeq were processed by \softwarename{minspec} (see section following), and hits not belonging to the minimal sets were removed.

\subsubsection{Identification of minimal species sets with \softwarename{minspec}}

A computational method to minimise false \ac{OTU} identifications and increase the accuracy of \ac{OTU} abundance estimates (\softwarename{minspec}) was developed and implemented in \softwarename{perl}\footnote{\softwarename{minspec} and the associated metagenomic simulation and validation scripts are open source and available at \url{https://github.com/wilkox/minspec}; a copy has also been provided in the supplementary information.}.
Following the approach of \citet{Ye:2009bl} to the parsimonious reconstruction of biochemical pathways (\softwarename{MinPath}), \softwarename{minspec} computes the smallest set of OTUs sufficient to explain a set of observed high-quality hits against RefSeq (or any other sequence database).
The minimal set computation is framed as a linear programming problem and solved with the \ac{GLPK} tool \ac{GLPSOL} (Free Software Foundation, Boston).
This approach eliminates many of the spurious \ac{OTU} identifications which result from reads with strong identity to more than one \ac{OTU}. 
The ``minimal species set'' is liable to exclude some low-abundance \acp{OTU}, but gives more faithful abundance estimates and eliminates many false positives.

To validate this approach and estimate error rates, simulated microbial assemblages were generated and simulated metagenomic sampling and \softwarename{blast} search was performed on each assemblage.
To simulate sequence identity between taxa, each simulated taxon went through up to fifty rounds in which another taxon was selected at random and deemed to have sequence identity with the first.
After each round, the this process was terminated with a 10\% probability to simulate an exponential curve of interrelatedness between taxa.
A random subset of the simulated taxa were then selected to form the simulated assemblage.
Combined with the simulated seqence identity between taxa, this caused some taxa in the assemblage to have identity to taxa outside it.
A simulated metagenomic sampling was then performed, in which a taxon was selected at random to generate a read.
To simulate a natural rank-abundance curve, the randomly selected taxon would be rejected with probability $1 - \frac{1}{ln(x)+1}$, where $x$ is the taxon's rank.
Simulated \softwarename{blast} matches to the taxon were generated for the remaining reads.
Each time a taxon was selected to produce a read, other taxa with simulated sequence identity were also randomly selected to produce \softwarename{blast} matches for that read, simulating the problem of a single read producing multiple matches to closely related taxa.

To fully explore the limits and reliability of \softwarename{minspec}, the simulated metagenomic experiment described above was performed with all possible permutations of the following parameters: number of simulated taxa [100; 1,000; 10,000; 50,000; 100,000]; size of simulated assemblage [1; 10; 100; 300; 500; 1,000; 10,000]; number of simulated metagenomic reads [10; 100; 1,000; 10,000; 100,000; 200,000; 500,000].
Each permutation was repeated five times, except for those where the size of the assemblage would exceed the number of taxa simulated.
The resulting simulated \softwarename{blast} outputs were processed with \softwarename{minspec}, and the false positive (percentage of taxa not in the assemblage which nevertheless survived \softwarename{minspec} filtering) and false negative (percentage of taxa present in the assemblage which were not present after minspec filtering) rates calculated.
Because a high false negative rate can arise from undersampling, a problem in metagenomic studies both real and simulated, an additional ``false negative (\softwarename{minspec})'' metric was calculated, which excluded taxa which were present in the assemblage but through random chance did not generate any reads, the equivilant of ``unsampled rare taxa''.
This rate thus represented only false negatives attributable to \softwarename{minspec} itself.
Finally, as a measure of \softwarename{minspec}'s usefulness, the proportion of ``false'' taxa --- those which generated \softwarename{blast} matches but were not part of the assemblage --- that were succesfully removed by \softwarename{minspec} was calulated.

\subsubsection{\ac{OTU} abundances and variance between zones}

The relative \ac{OTU} abundances for each sample were determined using the \softwarename{perl} script \ac{GAAS} \cite{Angly:2009ip}.
Briefly, \ac{GAAS} estimates the relative abundance of \acp{OTU} from the number and quality of \softwarename{blast} matches to each species, taking into account differences in genome size. 
\ac{GAAS} was run with the default settings. 
To normalise for reads which did not yield acceptable matches, the relative abundances for each sample were scaled by that sample's effective \softwarename{blast} hit rate. 
An \ac{OTU} profile was generated for each sample by encoding the scaled relative abundance of each \ac{OTU} from each size fraction as a separate variable.

To test the hypothesis that the oceanic zones harbour significantly different communities, \ac{ANOSIM} with 999 permutations was performed on a standardised, log-transformed Bray-Curtis resemblance matrix of \ac{OTU} profiles.
\ac{SIMPER} analysis was performed to identify the contribution of individual \acp{OTU} to differences between the zones. 
All statistical procedures were performed in \softwarename{PRIMER 6} as described by \citet{Clarke:2001ut}.

\subsection{Functional analysis of metagenomic data}

\subsubsection{\softwarename{blast} comparison to \ac{KEGG} database}

In order to identify functional differences between the zones, the set of metagenomic reads from each sample was compared against the \ac{KEGG} GENES database (retrieved July 2 2010 from \url{ftp://ftp.genome.jp/pub/kegg/genes/fasta/genes.pep}) with \softwarename{blastx}, with default parameters except for: maximum number of database sequence alignments 10; E-value threshold $1.0\times10^{-3}$; gap opening penalty 11; gap extension penalty 1; masking of query sequence by \softwarename{SEG} masking for lookup table only.

\subsubsection{Analysis of functional potential}

Genes identified by \softwarename{blastx} were aggregated to \ac{KEGG} ortholog groups according to the \ac{KEGG} Orthology schema (\url{ftp://ftp.genome.jp/pub/kegg/genes/ko}, retrieved Mar 29 2011), and ortholog group abundances calculated for each sample. 
Following \citet{Coleman:2010jj}, a read was considered a hit to a given ortholog group if the top three hits for that read (or all hits if fewer than three total hits) were to genes from the same ortholog group, and had bit scores \textgreater{} 40. 
If the bit score difference between any two top hits was greater than 30, only the hits above this difference were considered.

Ortholog group counts were then used to calculate the abundance of KEGG modules.
Because many ortholog groups are members of more than one module, the abundance $a_m$ of each module $m$ was calculated as 
\[
a_{m}=\sum_{K=1}^{n}\frac{C_{K}}{M_K}
\]
where $n$ is the number of ortholog groups $K$ belonging to module $m$, $C_{K}$ is the number of hits to ortholog group $K$, and $M_{K}$ is the total number of modules to which $K$ belongs.
To account for differences in sequencing depth between samples, module abundances were scaled to 500,000 reads per sample. 
To test the hypothesis that the \ac{NZ} and \ac{SZ} harbour significantly different functional potential, one-way \ac{ANOSIM} with 999 permutations was performed as above on a standardised, log-transformed Bray-Curtis distance resemblance matrix of the module and ortholog group profiles. 
A functional profile was generated for each sample by summing the scaled abundances of each module from all size fractions, and \ac{SIMPER} performed as above to identify modules which contributed highly to the variation in functional potential between the two zones. 
Modules with a high contribution to variance or otherwise of interest were then linked to taxonomy (``taxonomic decomposition'') by noting the genus of the organism associated with each gene in the \ac{KEGG} GENES database and thus calculating the relative contribution of each genus to each module's abundance. 
This allowed functional contributions to be putatively assigned to genera which were not identified in our taxonomic analysis, as the database included gene sequences for organisms for which a full genome was not available.

\section{Results}

\subsection{Metagenomic sequencing}
6.6 Gbp of 454 sequence data representing picoplankton in the size range 0.1 -- 3.0 \micron{} was obtained from 16 samples. 
After removal of low-quality reads, 454 sequencing yielded 157,507 -- 597,689 reads per sample (mean 354,399) of lengths ranging from 100 to 606 bp (mean 378).

\subsection{Phylogenetic analysis of metagenomic data}

The proportion of reads in each sample which yielded matches to RefSeq ranged from 25\% to 85\% (mean 62\%).
The most abundant \acp{OTU} in each sample are given in \tabreft{tab:topotus} and a full list of \ac{OTU} abundances in the supplementary material \suppfile{PF-all-OTUs.csv}.
All samples and size fractions exhibited very low \ac{OTU} evenness \figref{fig:rankabundance}.

\begin{sidewaystable}
\caption[Twenty most abundant \acp{OTU}]{Relative abundances (as percentages) of the twenty most abundant \acp{OTU} identified in this study, in each zone and size fraction.}
\label{tab:topotus}
\smallskip
\begin{tabularx}{\textheight}{Xlllllllll}
\toprule
OTU & \multicolumn{3}{c}{North} & \multicolumn{3}{c}{South} & \multicolumn{3}{c}{Deep}\\
\cmidrule(r){2-4}
\cmidrule(r){5-7}
\cmidrule(r){8-10}
& 0.1 \micron & 0.8 \micron & 3.0 \micron & 0.1 \micron & 0.8 \micron & 3.0 \micron & 0.1 \micron & 0.8 \micron & 3.0 \micron\\
\midrule

\emph{Candidatus} 'Pelagibacter ubique' HTCC1062 & 61.76 & 25.00 & 23.87 & 58.85 & 22.40 & 17.61 & 37.05 & 24.56 & 17.66\\
\emph{Nitrosopumilus maritimus} SCM1 & 0.01996 & 0.01438 & 0.009508 & 1.076 & 1.309 & 1.210 & 19.09 & 9.463 & 17.77\\
\emph{Candidatus} 'Ruthia magnifica' str. Cm (\emph{Calyptogena magnific}a) & 0.6699 & 0.6458 & 0.5484 & 2.987 & 2.616 & 1.025 & 3.945 & 4.601 & 2.264\\
\emph{Roseobacter} sp. OCh114 & 0.3125 & 2.932 & 1.588 & 0.4477 & 3.994 & 2.657 & 0.1259 & 1.228 & 0.6792\\
\emph{Synechococcus} sp. CC9902 & 0.1081 & 9.837 & 4.973 & 0.0007484 & 0.004156 & 0.09733 & 0.002846 & 0.01502 & 0.01058\\
\emph{Silicibacter pomeroyi} DSS-3 & 0.2578 & 2.286 & 1.154 & 0.3070 & 2.505 & 1.576 & 0.1224 & 0.9417 & 0.4988\\
\emph{Gramella forsetii} strain KT0803 & 0.2412 & 1.210 & 1.755 & 0.4993 & 2.347 & 1.890 & 0.2078 & 0.6179 & 0.5173\\
\emph{Candidatus} 'Vesicomyosocius okutanii' strain HA & 0.4634 & 0.4642 & 0.2078 & 1.970 & 1.807 & 0.2174 & 2.480 & 2.662 & 1.167\\
\emph{Robiginitalea biformata} strain HTCC2501 & 0.2751 & 1.099 & 1.297 & 0.4722 & 1.878 & 1.405 & 0.2265 & 0.6188 & 0.6946\\
\emph{Flavobacterium psychrophilum} strain JIP02/86 & 0.1718 & 0.8409 & 1.224 & 0.4316 & 1.960 & 1.598 & 0.1599 & 0.4744 & 0.6001\\
\emph{Synechococcus} sp. CC9311 & 0.03014 & 4.624 & 4.409 & 0.0007221 & 0.002778 & 0.02764 & 0.001580 & 0.002863 & 0.009241\\
\emph{Candidatus} 'Puniceispirillum marinum' IMCC1322 & 0.6444 & 2.077 & 1.267 & 0.3586 & 1.377 & 0.7109 & 0.3425 & 1.062 & 0.5345\\
\emph{Silicibacter} sp. TM1040 & 0.2274 & 1.652 & 0.8738 & 0.2709 & 1.803 & 1.233 & 0.07665 & 0.5890 & 0.2957\\
\emph{Jannaschia} sp. DFL-12 & 0.1776 & 1.378 & 0.7350 & 0.2443 & 1.692 & 0.8009 & 0.07338 & 0.6515 & 0.3078\\
\emph{Zunongwangia profunda} strain SM-A87 & 0.1522 & 0.7487 & 1.059 & 0.2968 & 1.410 & 1.204 & 0.1353 & 0.3478 & 0.4971\\
\emph{Colwellia} sp. 34H & 0.02345 & 0.3636 & 2.736 & 0.05207 & 0.5140 & 1.041 & 0.05137 & 0.4687 & 0.8013\\
\emph{Coraliomargarita akajimensis} strain DSM 45221 & 0.03698 & 0.07573 & 0.1197 & 0.1154 & 1.543 & 1.680 & 0.02614 & 0.3040 & 0.2740\\
\emph{Jannaschina} sp. CCS1 & 0.1173 & 0.9344 & 0.4784 & 0.1711 & 1.230 & 0.8239 & 0.05865 & 0.4462 & 0.2118\\
\emph{Pseudoalteromonas atlantica} strain T6c & 0.01251 & 0.4772 & 1.993 & 0.02270 & 0.4089 & 1.132 & 0.02634 & 0.2143 & 0.7459\\
\emph{Saccharophagus degradans} strain 2-40 & 0.06532 & 0.4325 & 0.5429 & 0.1289 & 1.072 & 0.8663 & 0.07798 & 0.2844 & 0.3165\\
\emph{Flavobacterium johnsoniae} strain UW101 & 0.08822 & 0.4220 & 0.6141 & 0.2034 & 0.9389 & 0.8578 & 0.07545 & 0.2255 & 0.3300\\
\emph{Capnocytophaga ochracea} strain DSM 7271 & 0.1143 & 0.4830 & 0.5399 & 0.2314 & 0.8815 & 0.6814 & 0.08964 & 0.2840 & 0.5043\\
\emph{Marinomonas} sp. MWYL1 & 0.03777 & 0.2529 & 0.3026 & 0.1514 & 1.300 & 0.7006 & 0.07393 & 0.2439 & 0.2155\\
\emph{Cellvibrio japonicus} strain Ueda107 & 0.05884 & 0.3080 & 0.3231 & 0.1155 & 0.9917 & 0.4713 & 0.06774 & 0.2981 & 0.2549\\
\emph{Marinobacter hydrocarbonoclasticus} VT8 & 0.04093 & 0.2889 & 0.3883 & 0.08418 & 0.7195 & 0.3848 & 0.1250 & 0.6667 & 1.066\\
\emph{Pseudoalteromonas haloplanktis} strain TAC125 & 0.01389 & 0.2505 & 0.8896 & 0.03427 & 0.3561 & 0.6530 & 0.1092 & 1.203 & 0.1503\\
\emph{Teredinibacter turnerae} strain T7901 & 0.05665 & 0.3051 & 0.3081 & 0.1138 & 0.9174 & 0.5127 & 0.06558 & 0.2649 & 0.1885\\
\emph{Acinetobacter baumannii} strain SDF & 0.004886 & 0.007187 & 0.4073 & 0.006260 & 0.04218 & 1.459 & 0.004285 & 0.01229 & 0.3155\\

\bottomrule
\end{tabularx}
\end{sidewaystable}

% the sample map
\begin{figure}
  \centering
  \includegraphics[width=\textwidth]{../polarfront/rankabundance.png}
  \caption[Rank-abundance curves for OTUs in each zone and size fraction]{Rank-abundance curves for OTUs identified in each zone and size fraction. The dominance of a given OTU is calculated as its relative abundance as a percentage of the relative abundance of all identified OTUs. The x-axis is scaled logarithmically.}
  \label{fig:rankabundance}
\end{figure}


\ac{ANOSIM} analysis showed that the zones harbor significantly different microbial communities (R = 0.451, p < 0.004). 
\ac{SIMPER} was performed in order to identify the contribution of individual \acp{OTU} to the difference between the \ac{NZ} and \ac{SZ}. 
The results for the highest contributors are provided in \tabreft{tab:otussimper}, and are graphically summarised for all \acp{OTU} in \figreft{fig:taxotreemap}.

\begin{sidewaystable}
\begin{center}
\caption[Highest-contributing \acp{OTU} to the difference between the North and South zones]{ 
The thirty \acp{OTU} with the highest contributions to the difference between the \ac{NZ} and \ac{SZ}. 
  Abundances are zonal averages and have been standardised and log-transformed.
  As each \ac{OTU} on each size fraction was encoded as a seperate variable in the \ac{SIMPER} analysis, the size fraction is given after each \ac{OTU} name.
  }
\label{tab:otussimper}
\smallskip
\begin{tabular}{llll}
\toprule
OTU & Abundance & Abundance & Contribution to\\
& South & North & variance (\%)\\

\midrule
\emph{Synechococcus} sp. CC9311 0.8 \micron & 0.00 & 1.08 & 2.88\\
\emph{Synechococcus} sp. CC9902 0.8 \micron & 0.00 & 1.04 & 2.81\\
\emph{Synechococcus} sp. CC9311 3.0 \micron & 0.01 & 0.98 & 2.59\\
\emph{Synechococcus} sp. CC9902 3.0 \micron & 0.04 & 0.76 & 2.03\\
\emph{Pelagibacter ubique} HTCC1062 3.0 \micron & 1.97 & 2.40 & 1.97\\
\emph{Ruthia magnifica} str. Cm (\emph{Calyptogena magnifica}) 0.1 \micron & 0.82 & 0.25 & 1.57\\
\emph{Colwellia} sp. 34H 3.0 \micron & 0.34 & 0.66 & 1.32\\
\emph{Ruthia magnifica} str. Cm (\emph{Calyptogena magnifica}) 0.8 \micron & 0.74 & 0.25 & 1.32\\
\emph{Pelagibacter ubique} HTCC1062 0.8 \micron & 2.32 & 2.48 & 1.32\\
\emph{Candidatus} Vesicomyosocius okutanii strain HA 0.1 \micron & 0.62 & 0.18 & 1.20\\
\emph{Coraliomargarita akajimensis} strain DSM 45221 0.8 \micron & 0.48 & 0.04 & 1.13\\
\emph{Coraliomargarita akajimensis} strain DSM 45221 3.0 \micron & 0.49 & 0.06 & 1.10\\
\emph{Roseobacter} sp. OCh114 0.8 \micron & 1.01 & 0.81 & 1.08\\
\emph{Pseudoalteromonas atlantica} strain T6c 3.0 \micron & 0.38 & 0.54 & 1.08\\
\emph{Candidatus} Vesicomyosocius okutanii strain HA 0.8 \micron & 0.57 & 0.19 & 1.04\\
\emph{Acinetobacter baumannii} strain SDF 3.0 \micron & 0.45 & 0.18 & 0.95\\
\emph{Gramella forsetii} strain KT0803 0.8 \micron & 0.72 & 0.43 & 0.94\\
\emph{Marinomonas} sp. MWYL1 0.8 \micron & 0.46 & 0.11 & 0.92\\
\emph{Roseobacter} sp. OCh114 3.0 \micron & 0.76 & 0.54 & 0.91\\
\emph{Flavobacterium psychrophilum} strain JIP02/86 0.8 \micron & 0.63 & 0.32 & 0.89\\
\emph{Silicibacter pomeroyi} DSS-3 0.8 \micron & 0.75 & 0.69 & 0.86\\
\emph{Brachyspira hyodysenteriae} strain WA1 3.0 \micron & 0.47 & 0.19 & 0.84\\
\emph{Ruthia magnifica} str. Cm (\emph{Calyptogena magnifica}) 3.0 \micron & 0.34 & 0.21 & 0.82\\
\emph{Pseudoalteromonas haloplanktis} strain TAC125 3.0 \micron & 0.22 & 0.33 & 0.77\\
\emph{Robiginitalea biformata} strain HTCC2501 0.8 \micron & 0.61 & 0.40 & 0.74\\
\emph{Nitrosopumilus maritimus} SCM1 0.1 \micron & 0.27 & 0.01 & 0.72\\
\emph{Gramella forsetii} strain KT0803 3.0 \micron & 0.59 & 0.59 & 0.71\\
\emph{Lysinibacillus sphaericus} strain C3-41 3.0 \micron & 0.29 & 0.02 & 0.71\\
\emph{Nitrosopumilus maritimus} SCM1 0.8 \micron & 0.25 & 0.01 & 0.70\\
\emph{Silicibacter} sp. TM1040 0.8 \micron & 0.59 & 0.55 & 0.69\\

\bottomrule
\end{tabular}
\end{center}
\end{sidewaystable}

% the sample map
\begin{figure}
  \centering
  \includegraphics[width=\textwidth]{../polarfront/taxotreemap.png}
  \caption[Contribution of \acp{OTU} to variance between the North and South zones]{Contribution of OTUs to variance between North and South, and differential abundance of OTUs from each size fraction between the two zones.
Each coloured (red or blue) rectangle represents an OTU identified through analysis of BLAST matches between SO metagenome data and the RefSeq database.
The area of each rectangle as a proportion of the total plot area corresponds to that OTU's contribution to the total variance between the two zones.
The colour of each rectangle corresponds to difference in relative abundance of that OTU between the zones, with blue indicating a higher relative abundance south of the PF, and red a higher abundance north of the PF.
OTUs from clades or taxonomic ranks of interest have been grouped, with labels in bold and groups separated by gray lines. 
Groups and OTUs with a low contribution to variance which were not grouped are unlabeled.
OTUs from each size fraction have also been grouped, with labels in black outline and size fractions separated by thick black lines. 
The total contribution to variance of each size fraction is given as a percentage.}
  \label{fig:taxotreemap}
\end{figure}


The \ac{SIMPER} analysis found that no single \ac{OTU} contributed more than 2.9\% of variance and 74\% of variance was contributed by \acp{OTU} with a contribution less than 1\%. 
There was also a large difference in the contribution to variance of the three size fractions, with approximately 52\% of all variance contributed by \acp{OTU} from the 3.0 \micron{} fraction, 37\% by the 0.8 \micron{} fraction, and 9\% by the 0.1 \micron{} fraction.
Notably, \acp{OTU} within several taxonomic groups that had high contribution to variance covaried in their relative representation in the \ac{NZ} and \ac{SZ}.
For example, Bacteroidetes and GSO-EOSA-1 representatives were on average more abundant in the \ac{SZ}; while \emph{Prochlorococcus} and \emph{Synechococcus} spp., SAR11 and SAR116 were on average more abundant in the \ac{NZ} \figref{fig:taxotreemap}.
Some groups, such as the Alteromonadales, had variable relative representation depending on size fraction.

\subsubsection{Validation of \softwarename{minspec}}

Repeated simulated metagenomic experiments with a wide range of permutations of parameters showed that \softwarename{minspec} was reliable and able to substantially reduce the rate of false positive \ac{OTU} identifications, although its effectiveness varied with the parameters of the assemblage and metagenomic experiment.

\begin{figure}
\centering

\begin{tabular}{cc}

\begin{subfigure}[b]{0.5\textwidth}
\centering
\includegraphics[width=\textwidth]{../polarfront/falsenegative.png}
\caption{TODO false negative}
\label{fig:minspecvalidationfalsenegative}
\end{subfigure}%

&
%\quad %add desired spacing between images, e. g. ~, \quad, \qquad etc. 
%(or a blank line to force the subfigure onto a new line)

\begin{subfigure}[b]{0.5\textwidth}
\centering
\includegraphics[width=\textwidth]{../polarfront/falsepositive.png}
\caption{TODO false positive}
\label{fig:minspecvalidationfalsepositive}
\end{subfigure}

\\
\bigskip
\\
\bigskip
\\
\bigskip
\\
%\quad %add desired spacing between images, e. g. ~, \quad, \qquad etc. 
%(or a blank line to force the subfigure onto a new line)

\begin{subfigure}[b]{0.5\textwidth}
\centering
\includegraphics[width=\textwidth]{../polarfront/minspecfalsenegative.png}
\caption{TODO minspec false negative}
\label{fig:minspecvalidationminspecfalsenegative}
\end{subfigure}

&
%\quad %add desired spacing between images, e. g. ~, \quad, \qquad etc. 
%(or a blank line to force the subfigure onto a new line)

\begin{subfigure}[b]{0.5\textwidth}
\centering
\includegraphics[width=\textwidth]{../polarfront/falsetaxaremoved.png}
\caption{TODO minspec false taxa removed}
\label{fig:minspecvalidationfalsetaxaremoved}
\end{subfigure}
\\

\end{tabular}

\caption{TODO master caption}\label{fig:minspecvalidation}
\end{figure}


The false negative rate, or percentage of taxa in the assemblage which were absent from the \softwarename{blast} results following \softwarename{minspec} processing, was generally high, ranging from \textapprox{} 20\% under ideal conditions (a low assemblage / all taxa ratio, and 500,000-read metagenomic sample) to \textapprox{} 90\% in the worst case (a high assemblage / all taxa ratio and a small metagenomic sample) \figref{fig:minspecvalidationfalsenegative}.
The assemblage / all taxa ratio (hereafter referred to as ``assemblage ratio'') indicates the proportion of simulated taxa (``all taxa'') which was chosen to form the simulated assemblage.
A higher ratio means it is more likely on average that any randomly selected taxon is part of the assemblage, and thus that any individual failure to detect a taxon is incorrect.
This problem is mitigated with increasing the number of reads, as this makes it less likely that a given taxon would go undetected.
The extreme false negative rates, in some cases 100\%, represent extreme simulated scenarios (e.g. an assemblage of 1 taxon drawn from a pool of 100,000), and thus are unlikely to reflect real metagenomic studies.

Because the majority of false negatives are attributable to undersampling and failure of taxa to generate \softwarename{blast} hits --- properties the simulated metagenomic experiments share with real ones --- a second metric, the false negative (\softwarename{minspec}) rate, was calculated \figref{fig:minspecvalidationminspecfalsenegative}.
This is the proportion of taxa in the assemblage which generated \softwarename{blast} hits, but were incorrectly removed by \softwarename{minspec}.
This rate thus represents error attributable only to \softwarename{minspec}.
The false negative (\softwarename{minspec}) rate was generally low, ranging from \textapprox{} 0--1\% for low assemblage ratios, to \textapprox{} 15--20\% under high ratios.
Surprisingly, increasing the number of reads only slightly decreased the rate, at both low and high assemblage ratios.
This may be because \softwarename{minspec} requires only one read which has identity to a single taxon to ensure that taxon is not removed.

The false positive rate, or percentage of taxa not in the assemblage which nevertheless generated high-quality \softwarename{blast} matches that were not removed by \softwarename{minspec}, was generally \textapprox{} 0--5\% except for extremely small read sets and low assemblage ratios, where it reached as high as 60\% \figref{fig:minspecvalidationfalsepositive}.
These results reinforce the value of larger read sets, and show that once a modest metagenome size is reached (\textapprox{} 100,000 reads) very few false positives can be expected.

The proportion of false taxa removed was calculated to measure \softwarename{minspec}'s success at identifying and eliminating taxa which are not part of the sampled assemblage yet generate high-quality \softwarename{blast} matches.
This rate varied from 0--1 depending on the parameters of the assemblage \figref{fig:minspecvalidationfalsetaxaremoved}.
For simulations with a low assemblage ratio, the proportion was generally high ($> 0.6$), although there were simulated experiments with a low ratio where the proportion was low or zero.
However, in all simulations with an assemblage ratio of 1, the proportion was 0, and the regression indicated a generally inverse relationship between the ratio and the proportion of false taxa removed.
This is likely because in assemblages with a higher assemblage ratio, there are fewer false taxa to remove; in assemblages with a ratio of 1, there are none.
The high proportion of false taxa correctly identified in simulations with a low assemblage ratio is thus a good indication that \softwarename{minspec} is generally successful at identifying and removing false taxa, especially as this proportion far exceeds the false positive and false negative (\softwarename{minspec}) rates for comparable experiments.
As expected, increasing the number of reads improved \softwarename{minspec}'s accuracy.

Overall, the simulated experiments validated both the accuracy and usefulness of \softwarename{minspec} as a tool for reducing error in metagenomic studies.
It is worth noting that the assemblage ratio is not an inherent property of an assemblage, although it is limited by the assemblage's species richness.
Rather, it can be decreased, and thus the accuracy of the metagenomic experiment improved, by performing \softwarename{blast} searches against larger databases with finer taxonomic resolution.
These results thus reinforce the value of both large read sets and comprehensive reference databases in obtaining high-quality metagenomic results.

\subsection{Functional analysis of metagenomic data}

\ac{ANOSIM} analysis of the samples' \ac{KEGG} ortholog group and module profiles revealed that the zones had significantly different functional potential (ortholog group: R = 0.642, p < 0.001; module: R = 0.871, p < 0.001). 
\ac{SIMPER} was performed on the profiles in order to identify the specific functional differences between the zones. 
The highest-contributing modules are given in \tabreft{tab:modulessimper}, and a complete list in the supplementary material \suppfile{PF-modules-SIMPER.csv}.
The highest-contributing ortholog groups are given in \tabreft{tab:orthologsimper}, and a complete list in the supplementary material \suppfile{PF-ortholog-groups-SIMPER.csv}.
No single ortholog group or module contributed more than 2.2\% of the variance, indicating a complex and diverse pattern of functional differences. 
There was a strong trend for ortholog groups and modules with higher contributions to variance to be overrepresented in the \ac{NZ} in the 3.0 \micron{} fraction but the \ac{SZ} in the smaller fractions, indicating that the functional diversity of each zone was strongly segregated by size fraction.

\begin{landscape}
\begin{table}
\sffamily
\begin{center}
\caption[Contributions of KEGG modules to variance between the North and South zones]{\sffamily{}The thirty \ac{KEGG} modules with the highest contributions to the difference between the \ac{NZ} and \ac{SZ}.
Abundances are zonal averages and have been standardised and log-transformed.
}
\label{tab:modulessimper}
\begin{tabularx}{\linewidth}{Xlll}
\toprule
\textbf{\ac{KEGG} module} & \textbf{Abundance} & \textbf{Abundance} & \textbf{Contribution to}\\
& \textbf{South} & \textbf{North} & \textbf{variance (\%)}\\
\midrule
Photosystem II & 0.42 & 0.57 & 2.21\\
Complex I (NADH dehydrogenase), NADH dehydrogenase I/diaphorase subunit of the bidirectional hydrogenase & 0.01 & 0.24 & 1.80\\
Photosystem I & 0.43 & 0.34 & 1.70\\
Pyrimidine deoxyribonucleotide biosynthesis, CDP/CTP \textrightarrow{} dCDP/dCTP,dTDP/dTTP & 0.51 & 0.66 & 1.16\\
Histidine degradation, histidine \textrightarrow{} N-formiminoglutamate \textrightarrow{} glutamate & 0.42 & 0.31 & 1.14\\
Methionine salvage pathway & 0.29 & 0.43 & 1.14\\
sn-Glycerol 3-phosphate transport system & 0.29 & 0.16 & 1.11\\
Complex I (NADH dehydrogenase), NADH dehydrogenase I & 1.08 & 1.05 & 1.06\\
Branched-chain amino acid transport system & 0.79 & 0.83 & 0.96\\
Dipeptide transport system & 0.14 & 0.02 & 0.95\\
Adenine nucleotide biosynthesis, IMP \textrightarrow{} ADP/dADP,ATP/dATP & 0.62 & 0.74 & 0.95\\
Glycine betaine/proline transport system & 0.66 & 0.56 & 0.94\\
Sulfur reduction, sulfate \textrightarrow{} H2S & 0.54 & 0.44 & 0.91\\
Simple sugar transport system & 0.46 & 0.39 & 0.90\\
Peptides/nickel transport system & 0.99 & 0.98 & 0.89\\
Ribosome, eukaryotes & 0.26 & 0.27 & 0.89\\
Multiple sugar transport system & 0.55 & 0.55 & 0.86\\
Type II general secretion system & 0.21 & 0.21 & 0.82\\
Sulfonate/nitrate/taurine transport system & 0.45 & 0.37 & 0.82\\
Guanine nucleotide biosynthesis, IMP \textrightarrow{} GDP/dGDP,GTP/dGTP & 0.72 & 0.82 & 0.81\\
RNA polymerase II, eukaryotes & 0.11 & 0.20 & 0.76\\
Histidine biosynthesis, PRPP \textrightarrow{} histidine & 0.94 & 0.86 & 0.76\\
Putrescine transport system & 0.18 & 0.09 & 0.72\\
Leucine biosynthesis, pyruvate \textrightarrow{} 2-oxoisovalerate \textrightarrow{} leucine & 1.29 & 1.37 & 0.71\\
C5 isoprenoid biosynthesis, non-mevalonate pathway & 0.70 & 0.77 & 0.71\\
Leucine degradation, leucine \textrightarrow{} acetoacetate + acetyl-CoA & 0.64 & 0.59 & 0.71\\
Thiamine transport system & 0.13 & 0.05 & 0.69\\
Spliceosome, 35S U5-snRNP & 0.18 & 0.20 & 0.68\\
Cytochrome b6f complex & 0.14 & 0.12 & 0.67\\
Menaquinone biosynthesis, chorismate \textrightarrow{} menaquinone & 0.25 & 0.27 & 0.66\\
\bottomrule
\end{tabularx}
\end{center}
\end{table}
\end{landscape}

\begin{sidewaystable}
\begin{center}
\caption[Contributions of KEGG ortholog groups to variance between the North and South zones]{
The thirty \ac{KEGG} ortholog groups with the higest contribution to the difference between the \ac{NZ} and \ac{SZ}.
Abundances are zonal averages and have been standardised and log-transformed.
As each ortholog group on each size fraction was encoded as a seperate variable in the \ac{SIMPER} analysis, the size fraction is given after each ortholog group name.
}
\label{tab:orthologsimper}
\smallskip
\begin{tabularx}{\textwidth}{Xlll}
\toprule
OTU & Abundance & Abundance & Contribution to\\
& South & North & variance (\%)\\
\midrule
Hypothetical protein 3.0 \micron & 0.11 & 0.24 & 0.26\\
Hypothetical protein 0.8 \micron & 0.68 & 0.57 & 0.24\\
Ribonucleoside-diphosphate reductase alpha chain [EC:1.17.4.1] 0.8 \micron & 0.17 & 0.24 & 0.15\\
DNA polymerase III subunit alpha [EC:2.7.7.7] 0.8 \micron & 0.25 & 0.19 & 0.14\\
Hypothetical protein 0.1 \micron & 0.26 & 0.24 & 0.12\\
Proline dehydrogenase / delta 1-pyrroline-5-carboxylate 0.8 \micron & 0.10 & 0.04 & 0.12\\
Aminomethyltransferase [EC:2.1.2.10] 0.8 \micron & 0.25 & 0.19 & 0.12\\
Ribonucleoside-diphosphate reductase alpha chain [EC:1.17.4.1] 3.0 \micron & 0.02 & 0.08 & 0.12\\
Sarcosine oxidase, subunit alpha [EC:1.5.3.1] 0.8 \micron & 0.22 & 0.17 & 0.12\\
Integrator complex subunit 6 3.0 \micron & 0.07 & 0.05 & 0.11\\
Multicomponent Na$^{+}$:H$^{+}$ antiporter subunit D 0.8 \micron & 0.11 & 0.05 & 0.11\\
Glutamine synthetase [EC:6.3.1.2] 0.8 \micron & 0.24 & 0.19 & 0.11\\
Pyruvate dehydrogenase E1 component [EC:1.2.4.1] 0.8 \micron & 0.15 & 0.10 & 0.11\\
Cobaltochelatase CobN [EC:6.6.1.2] 0.8 \micron & 0.11 & 0.06 & 0.11\\
Formate dehydrogenase, alpha subunit [EC:1.2.1.2] 0.8 \micron & 0.15 & 0.10 & 0.11\\
DNA-directed RNA polymerase subunit beta [EC:2.7.7.6] 3.0 \micron & 0.03 & 0.08 & 0.11\\
Glutamate synthase (NADPH/NADH) large chain [EC:1.4.1.13 1.4.1.14] 0.8 \micron & 0.25 & 0.22 & 0.11\\
Dimethylglycine dehydrogenase [EC:1.5.99.2] 0.8 \micron & 0.17 & 0.14 & 0.11\\
Flagellin 0.8 \micron & 0.06 & 0.10 & 0.10\\
DNA-directed RNA polymerase subunit beta [EC:2.7.7.6] 3.0 \micron{}\footnote{Due to an error in the \ac{KEGG} database, this module is encoded twice.} & 0.03 & 0.08 & 0.10\\
Photosystem II PsbA protein 0.8 \micron & 0.01 & 0.06 & 0.09\\
Aldehyde dehydrogenase (NAD+) [EC:1.2.1.3] 0.8 \micron & 0.17 & 0.13 & 0.09\\
Glutamate synthase (NADPH/NADH) large chain [EC:1.4.1.13 1.4.1.14] 3.0 \micron & 0.02 & 0.07 & 0.09\\
Thymidylate synthase (FAD) [EC:2.1.1.148] 0.8 \micron & 0.02 & 0.06 & 0.09\\
Topoisomerase IV subunit A [EC:5.99.1.-] 0.8 \micron & 0.11 & 0.07 & 0.09\\
DNA mismatch repair protein MutS 0.8 \micron & 0.13 & 0.08 & 0.09\\
Glutamate dehydrogenase [EC:1.4.1.2] 0.8 \micron & 0.07 & 0.03 & 0.09\\
DNA polymerase I [EC:2.7.7.7] 0.1 \micron & 0.12 & 0.11 & 0.09\\
GTP-binding protein 0.8 \micron & 0.26 & 0.21 & 0.09\\
GTP-binding protein 3.0 \micron & 0.03 & 0.07 & 0.09\\
\bottomrule
\end{tabularx}
\end{center}
\end{sidewaystable}


\section{Discussion}

\subsection{Taxonomic groups differentiating the zones}
TODO working on this section

\subsubsection{GSO-EOSA-1}

The Gammaproteobacterial Sulfur Oxidizer-EOSA-1 (GSO-EOSA-1) cluster, represented in RefSeq by the \acp{OTU} ``\emph{Candidatus} Vesicomyosocius okutanii'' strain HA and ``\emph{Candidatus} Ruthia magnifica'' strain Cm. (\emph{Calyptogena magnifica}) \cite{Walsh:2009fja}, made a large contribution to variance between the \ac{NZ} and \ac{SZ}, with higher abundance in the \ac{SZ}: relative abundances of GSO-EOSA-1 in the \ac{SZ} were 5.2\%, 3.4\% and 0.25\% in the 0.1, 0.8 and 3.0 \micron{} size fractions respectively, compared to 1.1\%, 0.84\% and 0.30\% in the \ac{NZ} \tabref{tab:topotus}.
The contribution to variance of this group was highest in the 0.1 \micron{} size fraction, followed by 0.8 \micron{} and 3.0 \micron{} \tabref{tab:otussimper}.
This pattern most likely represents a small cell size and lack of association with particulate matter.

``\emph{Ca.} R. magnifica'' and ``\emph{Ca.} V. okutanii'' are chemoautotrophic endosymbionts of deep-sea bivalves \cite{Kuwahara:2007gf,Newton:2007fu} and are thus unlikely to be present in open ocean surface waters. 
However, GSO-EOSA-1 representative ARCTIC96BD-19 has recently been reported at high abundance in Antarctic coastal waters \cite{Ghiglione:2011ee,Grzymski:2012ej}.
The majority of 16S rRNA genes from this metagenome with best \softwarename{blastn} matches to ``\emph{Ca.} R. magnifica'' and ``\emph{Ca.} V. okutanii'' clustered with ARTIC96BD-19 in a neighbour-joining phylogenetic tree \figref{fig:GSO-EOSA-1tree}, indicating this is the dominant GSO-EOSA-1 representative. 
\begin{figure}[!ht]
  \centering
  \includegraphics[width=\textwidth]{../polarfront/GSO-EOSA-1tree.png}
  \caption[Tree of GSO-EOSA-1 related 16S rRNA genes]{
  Neighbour-joining tree of GSO-EOSA-1-like 16S rRNA gene sequences from the metagenomes in this study.
  Sequences labeled in black text are reads from the metagenomes.
  Red labels are 16S rRNA gene sequences from Gammaproteobacterial Sulfur Oxidizers (GSO) and other Gammaproteobacteria.
  The tree was constructed using \softwarename{arb} \cite{Ludwig:2004dg}.
  }
  \label{fig:GSO-EOSA-1tree}
\end{figure}
\clearpage

Single-cell genomic analysis of ARCTIC96BD-19 from global mesopelagic waters indicates the lineage is probably mixotrophic, able to couple carbon fixation to oxidation of reduced sulphur compounds as well as assimilate organic carbon \cite{Swan:2011hb}.
GSO-EOSA-1 cytochrome C oxidase (CoxII) has been identified in a winter metaproteome of Antarctic Peninsula coastal waters, suggesting the capacity for aerobic respiration \cite{Williams:2012bs}.
Taken together, this evidence suggests the GSO-EOSA-1 representative in Antarctic coastal waters is a versatile chemolithoautotroph capable of aerobic respiration.

It has been proposed that during the winter months, chemolithoautotrophy is dominant over photoautotrophy as the major carbon fixation input in \ac{AZ} waters due to the lack of available light, both from seasonal darkness and ice cover \cite{Grzymski:2012ej}.
The high relative abundance of GSO-EOSA-1 we detected in \ac{SZ} compared to \ac{NZ} waters may therefore represent the remnants of an annual winter increase in population in the marginal ice zone which does not occur in the open ocean.

\subsubsection{Ammonia-oxidizing Crenarchaeota}

\emph{Nitrosopumilus maritimus} SCM1 and \emph{Cenarchaeum symbiosum} are chemolithoautotrophic, nitrifying members of the \ac{MGI} \cite{Preston:1996vi,Walker:2010ww} and are the only representatives in the reference database of the \ac{AOA}.
The contribution of \acp{OTU} of \emph{C. symbiosum} to the \ac{AOA} signature was low.
As \emph{C. symbiosum} is a sponge symbiont \cite{Preston:1996vi} and given the poor representation of \ac{AOA} in RefSeq, it is likely this \ac{OTU} has attracted sequences originating from planktonic \ac{AOA} and \emph{C. symbiosum} itself is not present.
\ac{AOA} were moderate contributors to variance between the \ac{NZ} and \ac{SZ}, and were overrepresented in the \ac{SZ} in all size fractions \figref{fig:taxotreemap}.
As with the GSO-EOSA-1 cluster, \ac{MGI} have been proposed to be abundant chemolithoautotrophs and therefore major drivers of winter carbon fixation in Antarctic coastal waters \cite{Grzymski:2012ej,Williams:2012bs}.

Sample 353 had a particularly high relative abundance of \emph{N. maritimus} \acp{OTU} (7.5\% of the 0.1 \micron{} fraction; 0.8 \micron: 11\%; 3.0 \micron: 12\%).
This sample was taken closer to the Antarctic continent (3.7 km) than any other, in relatively shallow (180 m) waters 17.6 km from the Mertz Glacier.
The high abundance of ammonia oxidizers may reflect an input of ammonia from terrestrial sources (e.g.{} penguin guano), or resuspension of benthic sediments in which \ac{MGI} are abundant \cite{Bowman:2003fa} by near-shore turbulence and iceberg scouring.
Breakdown of water column stratification has been previously suggested as a cause of increased \ac{AOA} abundance in Antarctic coastal surface waters \cite{Kalanetra:2009bv}.

\subsubsection{Cyanobacteria}

\acp{OTU} of the cyanobacterial genera \emph{Prochlorococcus} and \emph{Synechococcus} were overrepresented in the \ac{NZ} in all size fractions \figref{fig:taxotreemap}.
The mean relative abundance of cyanobacteria in samples 367 and 368, the two northernmost samples, was strikingly higher than the mean abundance across all other samples in the \ac{NZ}.
\emph{Synechococcus} sp. CC9902 alone composed greater than 22\% of the 0.8 \micron{} fraction in these samples, consistent with \emph{Synechococcus} species' average cell diameter of approximately 0.9 \micron.
The high abundance of both cyanobacterial genera on the 3.0 \micron{} fraction has previously been reported \cite{Lauro:2010jna} and may be attributable to aggregation \cite{Lomas:2011bp}.

Samples 367 and 368 were separated from the other samples north of the \ac{PF} by the \ac{STF}.
While the \ac{STF} was not a significant boundary on the assemblage level, it may mark a significant biogeographical boundary for these cyanobacteria.
\emph{Synechococcus} and \emph{Prochlorococcus} together represent a large proportion of both phytoplankton abundance and carbon fixation in temperate and tropical waters, in many regions contributing more than half of total primary production \cite{Liu:1997ub,Liu:1998tk,Andre:1999uh}.
The role of the \ac{STF} in determining the latitudinal range of \emph{Synechococcus} and \emph{Prochlorococcus} is therefore important, as it will affect models of ocean productivity under changing climactic conditions, and warrants further investigation.
Despite the high abundance of cyanobacteria north of the \ac{STF}, they were also a significant feature of the \ac{SAZ}; for example, \emph{Synechococcus} sp. CC9902 composed 3--5\% of the 0.8 \micron{} fraction in \ac{SAZ} samples.

These results extend the latitudinal distribution of both \emph{Prochlorococcus} and \emph{Synechococcus} to include presence at very low abundance as far south as the Antarctic coast.
\emph{Prochlorococcus} have been reported to be restricted to tropical and subtropical waters within 40\textdegree{} of latitude \cite{Partensky:1999uf}, and to be a negligible \cite{Ghiglione:2011ee} or undetectable \cite{Grzymski:2012ej} component of marine picoplankton in Antarctic waters.
However, these findings are consistent with findings of a logarithmic relationship of cyanobacterial numbers with temperature, where cyanobacteria were found at concentrations of 103 -- 104 cells per litre even in the coldest waters, approximately four orders of magnitude less than in waters around Tasmania \cite{Marchant:1987wv}.
Cyanophage proteins have also been detected in a metaproteomic analysis of Antarctic Peninsula coastal surface waters \cite{Williams:2012bs}.

\subsubsection{SAR11 and SAR116 clades}

``\emph{Candidatus} Pelagibacter ubique'' HTCC1062 is a good representative of total SAR11 abundance in this study, as it is a member of the SAR11 phylotype which is most abundant in \ac{SO} waters \cite{Brown:2012gna}.
``\emph{Ca.} P. ubique'' HTCC1062 was the most abundant \ac{OTU} across all samples and fractions (\ac{NZ} average: 62\%, 25\% and 24\% of the 0.1 \micron{}, 0.8 \micron{} and 3.0 \micron{} fractions respectively; \ac{SZ}: 59\%, 22\% and 18\%) and one of the most significant contributors to variance between the \ac{NZ} and \ac{SZ}.
The high abundance of SAR11 in the 0.1 \micron{} fraction is consistent with the small size of SAR11 cells \cite{Rappe:2002wz}.
The higher representation in the \ac{NZ} may reflect the competitiveness of SAR11 members in regions with low \ac{DOC} concentrations due to low primary productivity \cite{Giovannoni:2005ib,Alonso:2006dj}, such as the \ac{HNLC} \ac{SAZ}.
Overall, these findings are consistent with reports that SAR11 is ubiquitous in the world's oceans \cite{Mary:2006wk,Carlson:2009cc} and more abundant north of the \ac{ACC} \cite{Giebel:2009hr}.

\acp{OTU} of ``\emph{Candidatus} Puniceispirillum marinum'' from the SAR116 clade were a moderate contributor to variance between the \ac{NZ} and \ac{SZ} with higher abundance in the \ac{NZ} \figref{fig:taxotreemap}.
A genomic analysis reported ``\emph{Ca.} P. marinum'' IMCC1322 to be a metabolic generalist with genes for aerobic CO fixation, C1 metabolism and a ``\emph{Ca.} P. ubique''-like \ac{DMSP} demethylase, suggesting SAR116 and SAR11 occupy similar ecological niches \cite{Oh:2010di}.
In the Scotia Sea, SAR116 abundance (determined using fluorescence in situ hybridisation) was reported to be higher in more productive waters where SAR11 numbers were lower \cite{Topping:2006ul}.
However, this analysis across an extended latitudinal transect indicates that overall SAR11 and SAR116 have similar biogeographic distributions.

\subsubsection{Bacteroidetes}

OTUs of the phylum Bacteroidetes, in particular members of the class Flavobacteria, were found to be abundant (\ac{NZ} average: 1.2\%, 5.0\% and 6.9\% of the 0.1 \micron{}, 0.8 \micron{} and 3.0 \micron{} fractions respectively; SZ: 2.3\%, 9.8\% and 9.1\%) and significant contributors to variance between the \ac{NZ} and \ac{SZ} \figref{fig:taxotreemap}.
Flavobacteria have been previously reported to compose the majority of both Bacteroidetes \cite{Murray:2007db} and total planktonic biomass \cite{Abell:2005ji} in the \ac{SO}, as well as being abundant in sea ice \cite{Brown:2001hh}.
As heterotrophic degraders of \ac{HMW} compounds in the form of both \ac{DOM} and \ac{POM} \cite{Kirchman:2002ub}, marine Flavobacteria are major components of marine aggregates \cite{Rath:1998wm,Crump:1999wo,Zhang:2007fb}.
The higher abundance of Flavobacteria \acp{OTU} on the 0.8 \micron{} and 3.0 \micron{} fractions indicates their association with particulate matter.
Similar size partitioning of \ac{SO} Flavobacteria has previously been reported \cite{Abell:2005ji}.

The higher abundance of \acp{OTU} of Flavobacteria in the \ac{SZ} may reflect an input of cells from melting sea ice \cite{Brown:2001hh}, the higher rates of primary productivity in the south, and the role of the Flavobacteria as degraders of \ac{HMW} \ac{DOM}.
Because deposition of marine snow is a major route for sequestration of fixed carbon in the ocean \citep[e.g.][]{Hessen:2004vq}, the Flavobacteria that associate with this particulate matter represent a remineralizing shunt, which would decrease carbon sequestration by this route.

\subsubsection{Rhodobacterales}

Members of the order Rhodobacterales were abundant (\ac{NZ} average: 1.2\%, 10\% and 5.5\% of the 0.1 \micron{}, 0.8 \micron{} and 3.0 \micron{} fractions respectively; \ac{SZ}: 1.6\%, 13\% and 7.9\%) and high contributors to variance, overrepresented in the \ac{SZ} on all size fractions.
As several members of the Roseobacter clade have been shown to have symbiotic relationships with marine eukaryotic algae \cite{Anonymous:2005hd,WagnerDobler:2006kb}, and their abundance in the \ac{SO} has previously been linked to phytoplankton blooms \cite{West:2008kc,Obernosterer:2011df}, it is likely that their overrepresentation in the \ac{SZ} is related to the higher density of phytoplankton in the \ac{AZ}.

\acp{OTU} of \emph{Roseobacter denitrificans} Och114 and \emph{Silicibacter pomeroyi} DSS-3 were consistently the most abundant Roseobacter clade representatives.
\emph{R. denitrificans} and \emph{S. pomeroyi} fall within a subclade of \ac{AAP} members of the Roseobacter clade \cite{Swingley:2007dm}.
These species have diverse mixotrophic metabolisms, with genomic and experimental evidence of photoheterotrophic respiration of organic carbon, fixation of \ce{CO_{2}}, oxidation of CO, oxidation of reduced sulfur compounds, and utilization of the abundant marine osmolyte \ac{DMSP} \cite{King:2003kc,Moran:2004ie,WagnerDobler:2006kb,Swingley:2007dm,Brinkhoff:2008do,Howard:2008hf}.
This metabolic diversity suggests a complex ecological role, particularly with respect to the capture and release of climatically active gases (\ce{CO_{2}}, CO, dimethylsulfide) involved in carbon and sulfur cycling.

\subsubsection{Alteromondales}

Members of the gammaproteobacterial order Alteromonadales were large contributors to variance.
Most \acp{OTU} were overrepresented in the \ac{SZ} but some were overrepresented in the \ac{NZ} on the 3.0 \micron{} fraction \figref{fig:taxotreemap}.
\emph{Colwellia psychrerythraea} 34H was one of the most abundant \acp{OTU} in the Alteromonadales that exhibited this distribution (\ac{NZ} average: 0.14\%, 2.2\% and 16\% of the 0.1 \micron{}, 0.8 \micron{} and 3.0 \micron{} fractions respectively; \ac{SZ}: 0.52\%, 5.1\% and 10\%).
\emph{C. psychrerythraea} 34H was isolated from Arctic sediment, grows well at low temperatures and secretes extracellular polysaccharides \cite{Huston:2000jr,Junge:2003kb,Methe:2005uf}.
Similar to other \emph{Colwellia} species grown under laboratory conditions, cells have widths of 0.4--0.8 \micron{} and lengths of 1.5--4.5 \micron{} \cite{Jung:2006fh}.
Growth temperature can have a major impact on cell morphology, enzyme secretion and global gene expression in psychrophiles \citep[e.g.][]{Feller:2003ir,Junge:2003kb,Williams:2011hy,Cavicchioli:2006bl,Campanaro:2011gj}.
Moreover, marine bacteria can alter their cell dimensions in response to nutrient flux \citep[e.g.][]{Kjelleberg:1987wp}.
It is therefore possible that the populations of Alteromonadales captured on the 3.0 \micron{} filters (overrepresented in the \ac{NZ}) had different physiological properties to those on the 0.1 and 0.8 \micron{} filters (overrepresented in the \ac{SZ}).

\subsubsection{Verrucomicrobia}

TODO begin pasted block

Two representatives of the phylum Verrucomicrobia, \emph{Coraliomargarita akajimensis} and \emph{Akkermansia} sp. Muc-30, were moderate contributors to variance and overrepresented in the \ac{SZ} \figref{fig:taxotreemap}.
Surprisingly given the small cell size of \emph{C. akajimensis} \cite{Yoon:2007ic}, its contribution to variance increased with size fraction.
A global survey reported a similar fractionation pattern, and suggested marine Verrucomicrobia may be predominantly particle attached \cite{Freitas:2012jz}.
However, little else is known about the distribution and ecological roles of marine Verrucomicrobia \cite{Freitas:2012jz}.


TODO end pasted block

TODO up to here

\subsection{Functional capacities differentiating the zones}

\section{Conclusions}

