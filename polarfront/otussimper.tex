\begin{sidewaystable}
\sffamily
\begin{center}
\caption[Highest-contributing \acp{OTU} to the difference between the North and South zones]{\sffamily{}
The thirty \acp{OTU} with the highest contributions to the difference between the \ac{NZ} and \ac{SZ}. 
  Abundances are zonal averages and have been standardised and log-transformed.
  As each \ac{OTU} on each size fraction was encoded as a separate variable in the \ac{SIMPER} analysis, the size fraction is given after each \ac{OTU} name.
  }
\label{tab:otussimper}
\begin{tabular}{llll}
\toprule
\textbf{OTU} & \textbf{Abundance} & \textbf{Abundance} & \textbf{Contribution to}\\
& \textbf{South} & \textbf{North} & \textbf{variance (\%)}\\

\midrule
\genus{Synechococcus} sp. CC9311 0.8 \micron & 0.00 & 1.08 & 2.88\\
\genus{Synechococcus} sp. CC9902 0.8 \micron & 0.00 & 1.04 & 2.81\\
\genus{Synechococcus} sp. CC9311 3.0 \micron & 0.01 & 0.98 & 2.59\\
\genus{Synechococcus} sp. CC9902 3.0 \micron & 0.04 & 0.76 & 2.03\\
\candidatusfull{Pelagibacter ubique} HTCC1062 3.0 \micron & 1.97 & 2.40 & 1.97\\
\candidatusfull{Ruthia magnifica} str. Cm (\speciesfull{Calyptogena magnifica}) 0.1 \micron & 0.82 & 0.25 & 1.57\\
\genus{Colwellia} sp. 34H 3.0 \micron & 0.34 & 0.66 & 1.32\\
\candidatusfull{Ruthia magnifica} str. Cm (\speciesfull{Calyptogena magnifica}) 0.8 \micron & 0.74 & 0.25 & 1.32\\
\candidatusfull{Pelagibacter ubique} HTCC1062 0.8 \micron & 2.32 & 2.48 & 1.32\\
\candidatusfull{Vesicomyosocius okutanii} strain HA 0.1 \micron & 0.62 & 0.18 & 1.20\\
\speciesfull{Coraliomargarita akajimensis} strain DSM 45221 0.8 \micron & 0.48 & 0.04 & 1.13\\
\speciesfull{Coraliomargarita akajimensis} strain DSM 45221 3.0 \micron & 0.49 & 0.06 & 1.10\\
\genus{Roseobacter} sp. OCh 114 0.8 \micron & 1.01 & 0.81 & 1.08\\
\speciesfull{Pseudoalteromonas atlantica} strain T6c 3.0 \micron & 0.38 & 0.54 & 1.08\\
\candidatusfull{Vesicomyosocius okutanii} strain HA 0.8 \micron & 0.57 & 0.19 & 1.04\\
\speciesfull{Acinetobacter baumannii} strain SDF 3.0 \micron & 0.45 & 0.18 & 0.95\\
\speciesfull{Gramella forsetii} strain KT0803 0.8 \micron & 0.72 & 0.43 & 0.94\\
\genus{Marinomonas} sp. MWYL1 0.8 \micron & 0.46 & 0.11 & 0.92\\
\genus{Roseobacter} sp. OCh 114 3.0 \micron & 0.76 & 0.54 & 0.91\\
\speciesfull{Flavobacterium psychrophilum} strain JIP02/86 0.8 \micron & 0.63 & 0.32 & 0.89\\
\speciesfull{Silicibacter pomeroyi} DSS-3 0.8 \micron & 0.75 & 0.69 & 0.86\\
\speciesfull{Brachyspira hyodysenteriae} strain WA1 3.0 \micron & 0.47 & 0.19 & 0.84\\
\candidatusfull{Ruthia magnifica} str. Cm (\speciesfull{Calyptogena magnifica}) 3.0 \micron & 0.34 & 0.21 & 0.82\\
\speciesfull{Pseudoalteromonas haloplanktis} strain TAC125 3.0 \micron & 0.22 & 0.33 & 0.77\\
\speciesfull{Robiginitalea biformata} strain HTCC2501 0.8 \micron & 0.61 & 0.40 & 0.74\\
\speciesfull{Nitrosopumilus maritimus} SCM1 0.1 \micron & 0.27 & 0.01 & 0.72\\
\speciesfull{Gramella forsetii} strain KT0803 3.0 \micron & 0.59 & 0.59 & 0.71\\
\speciesfull{Lysinibacillus sphaericus} strain C3-41 3.0 \micron & 0.29 & 0.02 & 0.71\\
\speciesfull{Nitrosopumilus maritimus} SCM1 0.8 \micron & 0.25 & 0.01 & 0.70\\
\speciesfull{Silicibacter} sp. TM1040 0.8 \micron & 0.59 & 0.55 & 0.69\\

\bottomrule
\end{tabular}
\end{center}
\end{sidewaystable}
