\begin{table}
\begin{center}
\caption[TODO]{TODO Abundances are zonal averages and have been standardises and log-transformed.}
\label{tab:otussimper}
\begin{tabularx}{\textwidth}{Xlll}
\toprule
OTU & South & North & Contribution to\\
& & & variance (\%)\\

\midrule
\emph{Synechococcus} sp. CC9311 0.8 \micron & 0 & 1.08 & 2.88\\
\emph{Synechococcus} sp. CC9902 0.8 \micron & 0 & 1.04 & 2.81\\
\emph{Synechococcus} sp. CC9311 3.0 \micron & 0.01 & 0.98 & 2.59\\
\emph{Synechococcus} sp. CC9902 3.0 \micron & 0.04 & 0.76 & 2.03\\
\emph{Pelagibacter ubique} HTCC1062 3.0 \micron & 1.97 & 2.4 & 1.97\\
\emph{Ruthia magnifica} str. Cm (\emph{Calyptogena magnifica}) 0.1 \micron & 0.82 & 0.25 & 1.57\\
\emph{Colwellia} sp. 34H 3.0 \micron & 0.34 & 0.66 & 1.32\\
\emph{Ruthia magnifica} str. Cm (\emph{Calyptogena magnifica}) 0.8 \micron & 0.74 & 0.25 & 1.32\\
\emph{Pelagibacter ubique} HTCC1062 0.8 \micron & 2.32 & 2.48 & 1.32\\
\emph{Candidatus} Vesicomyosocius okutanii strain HA 0.1 \micron & 0.62 & 0.18 & 1.2\\
\emph{Coraliomargarita akajimensis} strain DSM 45221 0.8 \micron & 0.48 & 0.04 & 1.13\\
\emph{Coraliomargarita akajimensis} strain DSM 45221 3.0 \micron & 0.49 & 0.06 & 1.1\\
\emph{Roseobacter} sp. OCh114 0.8 \micron & 1.01 & 0.81 & 1.08\\
\emph{Pseudoalteromonas atlantica} strain T6c 3.0 \micron & 0.38 & 0.54 & 1.08\\
\emph{Candidatus} Vesicomyosocius okutanii strain HA 0.8 \micron & 0.57 & 0.19 & 1.04\\
\emph{Acinetobacter baumannii} strain SDF 3.0 \micron & 0.45 & 0.18 & 0.95\\
\emph{Gramella forsetii} strain KT0803 0.8 \micron & 0.72 & 0.43 & 0.94\\
\emph{Marinomonas} sp. MWYL1 0.8 \micron & 0.46 & 0.11 & 0.92\\
\emph{Roseobacter} sp. OCh114 3.0 \micron & 0.76 & 0.54 & 0.91\\
\emph{Flavobacterium psychrophilum} strain JIP02/86 0.8 \micron & 0.63 & 0.32 & 0.89\\
\emph{Silicibacter pomeroyi} DSS 3.- \micron08 & 0.75 & 0.69 & 0.86\\
\emph{Brachyspira hyodysenteriae} strain WA1 3.0 \micron & 0.47 & 0.19 & 0.84\\
\emph{Ruthia magnifica} str. Cm (\emph{Calyptogena magnifica}) 3.0 \micron & 0.34 & 0.21 & 0.82\\
\emph{Pseudoalteromonas haloplanktis} strain TAC125 3.0 \micron & 0.22 & 0.33 & 0.77\\
\emph{Robiginitalea biformata} strain HTCC2501 0.8 \micron & 0.61 & 0.4 & 0.74\\
\emph{Nitrosopumilus maritimus} SCM1 0.1 \micron & 0.27 & 0.01 & 0.72\\
\emph{Gramella forsetii} strain KT0803 3.0 \micron & 0.59 & 0.59 & 0.71\\
\emph{Lysinibacillus sphaericus} strain C3 4.1 \micron-30 & 0.29 & 0.02 & 0.71\\
\emph{Nitrosopumilus maritimus} SCM1 0.8 \micron & 0.25 & 0.01 & 0.7\\
\emph{Silicibacter} sp. TM1040 0.8 \micron & 0.59 & 0.55 & 0.69\\

\bottomrule
\end{tabularx}
\end{center}
\end{table}
