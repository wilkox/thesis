\begin{sidewaystable}
\caption[Twenty most abundant \acp{OTU}]{Relative abundances (as percentages) of the twenty most abundandent \acp{OTU} identified in this study.}
\label{tab:topotus}
\smallskip
\begin{tabularx}{\textheight}{Xlllllllll}
\toprule
& \multicolumn{3}{c}{North} & \multicolumn{3}{c}{South} & \multicolumn{3}{c}{Deep}\\
\cmidrule(r){2-4}
\cmidrule(r){5-7}
\cmidrule(r){8-10}
& 0.1 \micron & 0.8 \micron & 3.0 \micron & 0.1 \micron & 0.8 \micron & 3.0 \micron & 0.1 \micron & 0.8 \micron & 3.0 \micron\\
\midrule

\emph{Pelagibacter ubique} HTCC1062 & 61.76 & 25.00 & 23.87 & 58.85 & 22.40 & 17.61 & 37.05 & 24.56 & 17.66\\
\emph{Nitrosopumilus maritimus} SCM1 & 0.01996 & 0.01438 & 0.009508 & 1.076 & 1.309 & 1.210 & 19.09 & 9.463 & 17.77\\
\emph{Ruthia magnifica} str. Cm (\emph{Calyptogena magnific}a) & 0.6699 & 0.6458 & 0.5484 & 2.987 & 2.616 & 1.025 & 3.945 & 4.601 & 2.264\\
\emph{Roseobacter} sp. OCh114 & 0.3125 & 2.932 & 1.588 & 0.4477 & 3.994 & 2.657 & 0.1259 & 1.228 & 0.6792\\
\emph{Synechococcus} sp. CC9902 & 0.1081 & 9.837 & 4.973 & 0.0007484 & 0.004156 & 0.09733 & 0.002846 & 0.01502 & 0.01058\\
\emph{Silicibacter pomeroyi} DSS-3 & 0.2578 & 2.286 & 1.154 & 0.3070 & 2.505 & 1.576 & 0.1224 & 0.9417 & 0.4988\\
\emph{Gramella forsetii} strain KT0803 & 0.2412 & 1.210 & 1.755 & 0.4993 & 2.347 & 1.890 & 0.2078 & 0.6179 & 0.5173\\
\emph{Candidatus} Vesicomyosocius okutanii strain HA & 0.4634 & 0.4642 & 0.2078 & 1.970 & 1.807 & 0.2174 & 2.480 & 2.662 & 1.167\\
\emph{Robiginitalea biformata} strain HTCC2501 & 0.2751 & 1.099 & 1.297 & 0.4722 & 1.878 & 1.405 & 0.2265 & 0.6188 & 0.6946\\
\emph{Flavobacterium psychrophilum} strain JIP02/86 & 0.1718 & 0.8409 & 1.224 & 0.4316 & 1.960 & 1.598 & 0.1599 & 0.4744 & 0.6001\\
\emph{Synechococcus} sp. CC9311 & 0.03014 & 4.624 & 4.409 & 0.0007221 & 0.002778 & 0.02764 & 0.001580 & 0.002863 & 0.009241\\
\emph{Candidatus} Puniceispirillum marinum IMCC1322 & 0.6444 & 2.077 & 1.267 & 0.3586 & 1.377 & 0.7109 & 0.3425 & 1.062 & 0.5345\\
\emph{Silicibacter} sp. TM1040 & 0.2274 & 1.652 & 0.8738 & 0.2709 & 1.803 & 1.233 & 0.07665 & 0.5890 & 0.2957\\
\emph{Jannaschia} sp. DFL-12 & 0.1776 & 1.378 & 0.7350 & 0.2443 & 1.692 & 0.8009 & 0.07338 & 0.6515 & 0.3078\\
\emph{Zunongwangia profunda} strain SM-A87 & 0.1522 & 0.7487 & 1.059 & 0.2968 & 1.410 & 1.204 & 0.1353 & 0.3478 & 0.4971\\
\emph{Colwellia} sp. 34H & 0.02345 & 0.3636 & 2.736 & 0.05207 & 0.5140 & 1.041 & 0.05137 & 0.4687 & 0.8013\\
\emph{Coraliomargarita akajimensis} strain DSM 45221 & 0.03698 & 0.07573 & 0.1197 & 0.1154 & 1.543 & 1.680 & 0.02614 & 0.3040 & 0.2740\\
\emph{Jannaschina} sp. CCS1 & 0.1173 & 0.9344 & 0.4784 & 0.1711 & 1.230 & 0.8239 & 0.05865 & 0.4462 & 0.2118\\
\emph{Pseudoalteromonas atlantica} strain T6c & 0.01251 & 0.4772 & 1.993 & 0.02270 & 0.4089 & 1.132 & 0.02634 & 0.2143 & 0.7459\\
\emph{Saccharophagus degradans} strain 2-40 & 0.06532 & 0.4325 & 0.5429 & 0.1289 & 1.072 & 0.8663 & 0.07798 & 0.2844 & 0.3165\\
\emph{Flavobacterium johnsoniae} strain UW101 & 0.08822 & 0.4220 & 0.6141 & 0.2034 & 0.9389 & 0.8578 & 0.07545 & 0.2255 & 0.3300\\
\emph{Capnocytophaga ochracea} strain DSM 7271 & 0.1143 & 0.4830 & 0.5399 & 0.2314 & 0.8815 & 0.6814 & 0.08964 & 0.2840 & 0.5043\\
\emph{Marinomonas} sp. MWYL1 & 0.03777 & 0.2529 & 0.3026 & 0.1514 & 1.300 & 0.7006 & 0.07393 & 0.2439 & 0.2155\\
\emph{Cellvibrio japonicus} strain Ueda107 & 0.05884 & 0.3080 & 0.3231 & 0.1155 & 0.9917 & 0.4713 & 0.06774 & 0.2981 & 0.2549\\
\emph{Marinobacter hydrocarbonoclasticus} VT8 & 0.04093 & 0.2889 & 0.3883 & 0.08418 & 0.7195 & 0.3848 & 0.1250 & 0.6667 & 1.066\\
\emph{Pseudoalteromonas haloplanktis} strain TAC125 & 0.01389 & 0.2505 & 0.8896 & 0.03427 & 0.3561 & 0.6530 & 0.1092 & 1.203 & 0.1503\\
\emph{Teredinibacter turnerae} strain T7901 & 0.05665 & 0.3051 & 0.3081 & 0.1138 & 0.9174 & 0.5127 & 0.06558 & 0.2649 & 0.1885\\
\emph{Acinetobacter baumannii} strain SDF & 0.004886 & 0.007187 & 0.4073 & 0.006260 & 0.04218 & 1.459 & 0.004285 & 0.01229 & 0.3155\\

\bottomrule
\end{tabularx}
\end{sidewaystable}
