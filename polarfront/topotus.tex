\begin{sidewaystable}
\sffamily
\caption[Twenty most abundant \acp{OTU}]{\sffamily{}Relative abundances (as percentages) of the twenty most abundant \acp{OTU} identified in this study, in each zone and size fraction.}
\label{tab:topotus}
\begin{tabularx}{\textheight}{Xllllll}
\toprule
\textbf{OTU} & \multicolumn{3}{c}{\textbf{North}} & \multicolumn{3}{c}{\textbf{South}}\\
\cmidrule(r){2-4}
\cmidrule(r){5-7}
& 0.1 \micron & 0.8 \micron & 3.0 \micron & 0.1 \micron & 0.8 \micron & 3.0 \micron\\
\midrule

\candidatusfull{Pelagibacter ubique} HTCC1062 & 61.76 & 25.00 & 23.87 & 58.85 & 22.40 & 17.61\\
\speciesfull{Nitrosopumilus maritimus} SCM1 & 0.01996 & 0.01438 & 0.009508 & 1.076 & 1.309 & 1.210\\
\candidatusfull{Ruthia magnifica} str. Cm (\speciesfull{Calyptogena magnifica}) & 0.6699 & 0.6458 & 0.5484 & 2.987 & 2.616 & 1.025\\
\genus{Roseobacter} sp. OCh 114 & 0.3125 & 2.932 & 1.588 & 0.4477 & 3.994 & 2.657\\
\genus{Synechococcus} sp. CC9902 & 0.1081 & 9.837 & 4.973 & 0.0007484 & 0.004156 & 0.09733\\
\speciesfull{Silicibacter pomeroyi} DSS-3 & 0.2578 & 2.286 & 1.154 & 0.3070 & 2.505 & 1.576\\
\speciesfull{Gramella forsetii} strain KT0803 & 0.2412 & 1.210 & 1.755 & 0.4993 & 2.347 & 1.890\\
\candidatusfull{Vesicomyosocius okutanii} strain HA & 0.4634 & 0.4642 & 0.2078 & 1.970 & 1.807 & 0.2174\\
\speciesfull{Robiginitalea biformata} strain HTCC2501 & 0.2751 & 1.099 & 1.297 & 0.4722 & 1.878 & 1.405\\
\speciesfull{Flavobacterium psychrophilum} strain JIP02/86 & 0.1718 & 0.8409 & 1.224 & 0.4316 & 1.960 & 1.598\\
\genus{Synechococcus} sp. CC9311 & 0.03014 & 4.624 & 4.409 & 0.0007221 & 0.002778 & 0.02764\\
\candidatusfull{Puniceispirillum marinum} IMCC1322 & 0.6444 & 2.077 & 1.267 & 0.3586 & 1.377 & 0.7109\\
\genus{Silicibacter} sp. TM1040 & 0.2274 & 1.652 & 0.8738 & 0.2709 & 1.803 & 1.233\\
\genus{Jannaschia} sp. DFL-12 & 0.1776 & 1.378 & 0.7350 & 0.2443 & 1.692 & 0.8009\\
\speciesfull{Zunongwangia profunda} strain SM-A87 & 0.1522 & 0.7487 & 1.059 & 0.2968 & 1.410 & 1.204\\
\genus{Colwellia} sp. 34H & 0.02345 & 0.3636 & 2.736 & 0.05207 & 0.5140 & 1.041\\
\speciesfull{Coraliomargarita akajimensis} strain DSM 45221 & 0.03698 & 0.07573 & 0.1197 & 0.1154 & 1.543 & 1.680\\
\genus{Jannaschia} sp. CCS1 & 0.1173 & 0.9344 & 0.4784 & 0.1711 & 1.230 & 0.8239\\
\speciesfull{Pseudoalteromonas atlantica} strain T6c & 0.01251 & 0.4772 & 1.993 & 0.02270 & 0.4089 & 1.132\\
\speciesfull{Saccharophagus degradans} strain 2-40 & 0.06532 & 0.4325 & 0.5429 & 0.1289 & 1.072 & 0.8663\\
\speciesfull{Flavobacterium johnsoniae} strain UW101 & 0.08822 & 0.4220 & 0.6141 & 0.2034 & 0.9389 & 0.8578\\
\speciesfull{Capnocytophaga ochracea} strain DSM 7271 & 0.1143 & 0.4830 & 0.5399 & 0.2314 & 0.8815 & 0.6814\\
\genus{Marinomonas} sp. MWYL1 & 0.03777 & 0.2529 & 0.3026 & 0.1514 & 1.300 & 0.7006\\
\speciesfull{Cellvibrio japonicus} strain Ueda107 & 0.05884 & 0.3080 & 0.3231 & 0.1155 & 0.9917 & 0.4713\\
\speciesfull{Marinobacter hydrocarbonoclasticus} VT8 & 0.04093 & 0.2889 & 0.3883 & 0.08418 & 0.7195 & 0.3848\\
\speciesfull{Pseudoalteromonas haloplanktis} strain TAC125 & 0.01389 & 0.2505 & 0.8896 & 0.03427 & 0.3561 & 0.6530\\
\speciesfull{Teredinibacter turnerae} strain T7901 & 0.05665 & 0.3051 & 0.3081 & 0.1138 & 0.9174 & 0.5127\\
\speciesfull{Acinetobacter baumannii} strain SDF & 0.004886 & 0.007187 & 0.4073 & 0.006260 & 0.04218 & 1.459\\

\bottomrule
\end{tabularx}
\end{sidewaystable}
