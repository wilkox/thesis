\chapter{Introduction} 
\label{ch:intro}

Sections of this chapter have been previously published in \bibentry{Wilkins:2012ii}.

\section{Microbial ecology of the Southern Ocean}
TODO working on this

TODO begin pasted block

##Bacteria

###Alphaproteobacteria

####Roseobacter

The Roseobacter clade is an abundant and ecologically significant group of marine bacteria, found at high (> 15%) abundance in most marine surface environments (Buchan et al. 2005 and references therein). Unlike some other major proteobacterial groups which are strongly associated with a particular ecological niche (e.g. the SAR11 clade), roseobacters have diverse metabolic abilities, with members capable (for example) of aerobic anoxygenic phototrophy (Biebl et al. 2005; Béjà et al. 2002), degradation of dimethylsulfoniopropionate (DMSP) by at least two pathways (Miller and Belas 2004; Moran et al. 2003), carbon monoxide oxidation (King 2003) and heterotrophic utilisation of a broad range of substrates (reviewed in Brinkhoff et al. 2008). Roseobacters are found in the planktonic fraction as well as in commensal association with phytoplankton and metazoans (reviewed in Buchan et al. 2005).

Several 16S rDNA-based studies have identified the Roseobacter Clade Affiliated (RCA) subgroup as ubiquitous and abundant in Southern Ocean surface waters and to a depth of at least 2200 m, composing ~10-30% of surface bacteria (and the majority of Roseobacters) in the Subantarctic and Antarctic zones (Giebel et al. 2009; Murray and Grzymski 2007; Ghiglione and Murray 2011; Selje et al. 2004; Manganelli et al. 2009; Wilkins et al. ????) and a major fraction of the population in coastal waters (Murray and Grzymski 2007; Koh et al. 2011). Two major RCA phylotypes appear to be present in the SO and form the majority of the Roseobacter population. The phylotypes are strictly segregated by the Polar Front, coexisting only within the Polar Frontal Zone (Selje et al. 2004; Giebel et al. 2009) where they may outnumber even the SAR11 clade. There is some evidence that the AZ RCA phylotype originates from the North Atlantic; Giebel et al. (2009) noted Circumpolar Deep Water (CDW) at 2200m in the Subantarctic zone that the waters had an identical temperature-salinity signature to North Atlantic Deep Water (NADW). NADW is formed by the sinking of dense, saline waters in the surface north Atlantic, and is transported to the SO via global thermohaline circulation to become CDW (Callahan 1972). Consistent with the upwelling of CDW in the AZ south of the PF, Selje et al. (2004) reported in a global study of RCA 16S rDNA gene fragments that the surface phylotype south of the Polar Front was identical to one found in the Arctic Ocean, while differing by 3 bp from that north of the PF.

Little is known about the functional capabilities of RCA as only two isolated representatives have been described to date. Giebel et al. (2010) isolated Candidatus Planktomarina temperata from the North Sea, where it was the dominant phylotype. The authors' identification of the pufM gene encoding a bacteriochlorophyll a subunit suggests at least this member of the RCA is capable of performing aerobic anoxygenic photosynthesis, a function of potentially large ecological significance. Mayali et al. (2008) isolated an apparently heterotrophic RCA member from subtropical waters, and found in vitro evidence that they colonised and increased mortality in blooming dinoflagellates, but did not investigate photosynthetic potential.

Roseobacters, and particularly the RCA, have been strongly associated with phytoplankton blooms in the SO. Two separate 16S rDNA-based studies of a naturally fertilised bloom in the Kerguelen islands region (West et al. 2008; Obernosterer et al. 2011) found that RCA and the Roseobacter NAC11-7 and NAC11-6 clusters were dominant bacterial OTUs in the bloom patch, suggesting they play a role in heterotrophic degradation of bloom products. Unlike the other clusters, however, RCA representatives were also relatively abundant and metabolically active outside of the patch. Both Giebel et al. (2009) and Obernosterer et al. (2011) found that in SO vertical profiles RCA abundances often peaked at the deep chlorophyll maximum, again suggesting an association with phytoplankton.

RCA abundance may follow a seasonal cycle in the SO. Giebel et al. (2009) found that RCA phylotypes were at maximum 8% of all bacterial 16S rDNA genes during winter but up to 36% in the coastal current and Weddell sea during autumn, while Ghiglione and Murray (2011) found the proportion to peak in January in coastal waters off the Antarctic Peninsula and in February off the Kerguelen islands. 

A metagenomic study of Southern Ocean waters off West Antarctica found that Roseobacter clade SSU rRNA sequences were much more abundant in summer than in winter, with Sulfitobacter sequences the most abundant within this clade (Murray et al., 2012). This is consistent with the association of Roseobacters with phytoplankton (Moran et al., 2005). Nevertheless, Roseobacter clade representatives in these polar waters are metabolically active in both seasons, with an emphasis on high-affinity uptake systems (ABC, TRAP) for capturing labile nutrients such as sugars, polyamines, amino acids, and oligopeptides (Williams et al., 2012).


####SAR11

The SAR11 clade of alphaproteobacteria is probably the most abundant class of marine microorganisms worldwide (Morris et al. 2002). Pelagibacter ubique strain HTCC1062, the first and most intensively studied SAR11 isolate, has one of the smallest genomes and gene complements of any known free-living cell as well as a very small cell volume (Giovannoni et al. 2005). The small cell volume, streamlined genome and high proportion of ABC nutrient-uptake transporter genes are all consistent with an oligotrophic lifestyle, scavenging a wide range of substrates using high-affinity, broad-specificity transporters (Giovannoni et al. 2005; Lauro et al. 2009; Sowell et al. 2009). SAR11 cells probably preferentially utilise low over high molecular weight DOM (Malmstrom et al. 2005) and their relative contribution to uptake of DOM may decrease as substrate concentration increases (Alonso and Pernthaler 2006). A consequence of this oligotrophic strategy is that SAR11 members are probably unable to take advantage of sudden nutrient influxes, such as during phytoplankton blooms, to rapidly increase cell density (Tripp et al. 2008).

SAR11 has been consistently detected at high abundances in molecular surveys of the Southern Ocean, in all open ocean regions as well as at depth and in coastal waters, and is usually the dominant alpha-proteobacterial, if not bacterial, group (Giebel et al. 2009; Murray and Grzymski 2007; López-García et al. 2001; Straza et al. 2010; Jamieson et al. 2012; García-Martínez and Rodríguez-Valera 2000; Ghiglione and Murray 2011; Murray et al. 2010; Piquet et al. 2011). It is probably more abundant in the epipelagic than at depth (Giebel et al. 2009; Wilkins et al. ????).

As with the other major bacterial groups, SAR11 seems to exhibit biogeographic partitioning in the SO, and is probably represented by two major ecotypes with a temperature-driven boundry in the region of the PF (Brown et al. ????). It is probably more abundant in the Subantarctic and polar frontal zones than in the Antarctic Zone (Giebel et al. 2009; Wilkins et al. ????; Ghiglione and Murray 2011). This may be related to a competitive advantage of the oligotrophic SAR11 in the HNLC Subantarctic relative to the AZ, where blooming phytoplankton lead to increased concentrations of HMW DOM and POM. Straza et al. (2010) found SAR11 accounted for the largest fraction of leucine uptake among all bacterial groups in continental shelf waters off the West Antarctic Peninsula, but a comparatively small fraction of protein uptake, consistent with a role as a LWM DOM specialist. West et al. (2008), examining 16S rDNA profiles in and out of a natural phytoplankton bloom on the Kerguelen Plateau (Subantarctic), found SAR11 to be a dominant group in HNLC waters outside the bloom patch but relatively less abundant in it. A separate study of the same bloom found SAR11 had a markedly smaller relative contribution to bulk leucine incorporation in the patch than out, suggesting it was not a major contributor to DOM degradation (Obernosterer et al. 2011). Interestingly, SAR11 did dominate in abundance and leucine incorporation at an additional site where a recent and transient phytoplankton bloom had taken place, implying a time lag in the succession between the baseline HNLC and bloom populations. The authors additionally noted that SAR11 abundances at the bloom station began to climb towards non-bloom levels once the bloom had peaked and begun declining. An Antarctic Peninsula SAR11 metaproteome was dominated by ABC transport proteins for the capture and uptake of labile substrates, especially taurine, polyamines and amino acids, and also included DMSP demethylase (Williams et al., 2012). Finally, despite an apparently negative correlation between SAR11 and blooming phytoplankton, Ghiglioni and Murray (2011) found only small seasonal changes in abundance during an annual cycle at the AP and Kerguelen Island. These studies are all consistant with the view of SAR11 as a typically non-opportunistic oligotroph specialising in LMW DOC.

One of the most interesting physiological features of SAR11 representatives is their expression of the retinal-binding pigment proteorhodopsin, which has been shown to act as a proton pump when exposed to light (Béjà et al. 2000) and has therefore been implicated in photoheterotrophy. Surprisingly, given very low light levels in Antarctic waters during austral winter, SAR11 proteorhodopsin is present throughout the annual cycle (Williams et al., 2012). This may be consistent with the observation that many marine proteorhodopsins do not appear tuned to maximise energy conversion from available light, which has led Fuhrman et al. (2008) to propose at least some proteorhodopsins may perform non-energetic functions such as photoregulatory sensing. Alternatively, constitutive expression of proteorhopospin for light harvesting in SAR11 may facilitate the ability to immediately respond to cellular energy deficits caused by carbon starvation (Steindler et al., 2011).

####SAR116

The SAR116 clade of alphaproteobacteria have been detected throughout the world ocean. SAR116 has been detected in molecular studies of the SO, in both the SAZ (West et al. 2008; Topping et al. 2006) and at lower abundance in the AZ (Wilkins et al. ????). Estimates of its relative abundance vary depending on the method used; Topping et al. (2006) using FISH estimated it composed 13.1 +- 8.6% to 31.9 +- 13.7% of bacterioplankton in the West and East regions of the Scotia Sea respectively, while Wilkins et al. (????) estimated a relative abundance of 0.74% of the picoplanktonic fraction in the SAZ and 0.38% in the AZ in a metagenomic survey. 

The only isolated SAR116 representative, Candidatus Puniceispirillum marinum, has been reported to have a versatile repertoire of genes for aerobic CO fixation, C1 metabolism and dimethylsulfoniopropionate degradation, suggesting it may occupy a 'marine generalist' niche similar to that of SAR11 and some Roseobacters (Oh et al. 2010). Proteins for ABC and TRAP transport and C1 metabolism with high matches to SAR116 bacteria were detected in both the summer and winter metaproteomes of coastal waters of the Antarctic Peninsula, consistent with a preference for labile compounds and C1 substrates (Williams et al., 2012).

###Betaproteobacteria

The Betaproteobacteria are a large and cosmopolitan class with a range of ecological roles in the World Ocean (reviewed in Kirchman 2008). While not found at high abundance (Gentile et al. 2006; Ghiglione and Murray 2011; Jamieson et al. 2012), there is evidence that Betaproteobacteria perform significant ecological functions. Most known ammonium oxidizing bacteria (AOB) belong to the betaproteobacteria (Head et al. 1993; Teske et al. 1994). Hollibaugh et al. (2002) detected Nitrosospira-like 16S rRNA sequences in Ross Sea and Antarctic Peninsula surface waters, and noted that the ribotype appeared similar to one found in the Arctic. A metagenomic survey also detected Nitrosomonas europaea, Nitrosomonas eutropha and Nitrosospira multiformis strains in all SO surface waters, and at significantly higher abundances in Antarctic Bottom Water (Wilkins et al. ????). However, ammonia-oxidizing archaea outnumbered ammonia-oxidizing bacteria at most sites, consistent with the view that the former are the major nitrifiers in the marine environment (Wuchter et al. 2006). Metagenomic and metaproteomic analyses of surface coastal waters off the Antarctic Peninsula show evidence of Calvin cycle carbon fixation and ammonia oxidation in winter performed by ammonia-oxidizing Betaproteobacteria (Grzymski et al., 2012; Williams et al., 2012).

The OM43 clade of betaproteobacteria has been associated with coastal phytoplankton blooms (Morris et al. 2006) and shown to be an obligate methylotroph capable of utilising methanol and formaldehyde as carbon and energy sources (Giovannoni et al. 2008). As it has the smallest reported genome for a free-living cell, OM43 seems to be highly specialized for this unusual niche (the 'genome streamlining' hypothesis, Mira et al. (2001)). OM43 has been detected in a 16S rDNA library in a naturally fertilised bloom in the SAZ (West et al. 2008), where it was the only betaproteobacterial representative, and in a metaproteomic survey of coastal waters on the AP, where methanol dehydrogenase from OM43 was detected (Williams et al., 2012). Although the source of methanol in the marine environment is not yet clear, it may be a byproduct of phytoplankton growth (Heikes et al. 2002) which would be consistent with OM43's observed association with coastal blooms. This possibility suggests OM43, and perhaps other C1 specialists, play an underexplored role in the marine microbial loop. Alternative sources are atmospheric deposition (Sinha et al. 2007) or photochemical degradation of organic material (Dixon et al. 2011). The latter is of particular interest in Antarctic waters, given the high levels of solar irradiation during the austral summer.

###Gammaproteobacteria

####SAR86

The gammaproteobacterial SAR86 clade is an abundant group in the surface ocean, being e.g. the most abundant genome for an uncultured organism in the GOS dataset (Dupont et al. 2011). While it has been detected in the Southern Ocean (Abell and Bowman 2005; Topping et al. 2006; West et al. 2008; Obernosterer et al. 2011), little is known of its distribution or ecological role. Topping et al. (2006) estimated on the basis of FISH activity that SAR86 cells composed 7.8% +-8.2 and 18.3% +- 17.0 of total bacterioplankton in the western and eastern Scotia Sea respectively, suggesting that at least in the SAZ it is a major component of the surface community. Genomic analysis of partial SAR86 genomes assembled from metagenomic datasets found the clade have streamlined genomes and are specialized for utilizing lipids and carbohydrates, suggesting minimal competition between SAR86 and SAR11 for DOC (Dupont et al. 2011). This may be reflected by the simultaneous high abundance and activity of SAR11 and SAR86 in the HNLC waters of the SAZ (Obernosterer et al. 2011).

####OMG group

The term oligotrophic marine Gammaproteobacteria (OMG) was named for a group of physiologically diverse heterotrophs that belong to previously detected environmental rRNA clades (OM60, BD1-7, KI89A, OM182, SAR92) (Cho and Giovannoni, 2004). Cultured OMG isolates have been shown to be obligately oligotrophic (Cho and Giovannoni, 2004). Nevertheless, SAR92 is associated with nutrient-rich waters with high phytoplankton abundances (Stingl et al. 2007; Pinhassi et al. 2005). Reports of SAR92 in the SO corroborate this ecology: both West et al. (2008) and Obernosterer et al. (2011) found SAR92-affiliated OTUs to be far more abundant inside the KEOPS phytoplankton bloom patch than in typically HNLC SAZ waters outside of it, with abundance declining as the bloom aged. This, combined with the observation that SAR92 growth is highly carbon-limited (Stingl et al. 2007), suggests the clade plays an important role in degradation of organic carbon produced by phytoplankton blooms. It has also been detected in coastal AP and Kerguelen Islands waters (Ghiglione and Murray 2011). Metaproteomic and metagenomic surveys of coastal waters at Palmer station found OMG to me more abundant in the summer than winter (Williams et al., 2012). TonB-dependent receptor systems from OMG were highly abundant in the metaproteome, indicating that this is the preferred uptake system of ambient substrates (Williams et al., 2012). Certain OMG strains encode proteorhodopsin (HTCC2207, Stingl et al., 2007; HTCC2143, Oh et al., 2010), also indicated in the metaproteomic study in both seasons (Williams et al., 2012).



####Ant4D3

In a study of six fosmids from nearshore waters at Palmer station, Grzymski et al. (2006) identified a uncultured gammaproteobacterium, Ant4D3. It has since been reported as one of the dominant proteobacterial groups in the Southern Ocean. In waters off the western Antarctic Peninsula, Ant4D3 was reported to compose 10% of the total community and half the gammaproteobacterial community, and 68% of cells incorporating amino acids (Straza et al. 2010). The authors also reported that the clade appears to have low diversity, based on detected rDNA sequences. Like SAR86, Ant4D3 cells were more active in HNLC than bloom conditions on the Kerguelen Plateau (West et al. 2008). However, Ghiglione and Murray (2011) reported that 16.5% of tag-pyrosequenced 16S DGGE bands from summer AP waters were affiliated to Ant4D3, dominating the gammaproteobacteria and outnumbering winter and Kerguelen Island waters. Murray et al. (2010) similarly found Ant4D3 clones to be highly abundant in a 16S library from waters in the vicinity of Antarctic icebergs. Little is known about the group's function or ecological position, although it has been detected in Arctic waters where it appeared to occupy a DOM utilisation niche different from that of other major heterotrophs e.g. SAR11 (Nikrad et al. 2012).

####GSO-EOSA-1

The GSO-EOSA-1 cluster of sulfur-oxidizing gammaproteobacteria, which includes the uncultivated ARCTIC96BD-19 and SUP05 lineages as well as cultivated chemoautotrophic clam symbionts, has been reported in global mesopelagic waters (Swan et al. 2011) and oxygen minimum zones (Walsh et al. 2009; Canfield et al. 2010). Three studies have recently identified GSO-EOSA-1 representatives at high abundance in coastal and AZ waters. A metagenomic survey of coastal waters at Palmer station found GSO-EOSA-1
winter bacterioplankton were dominated by Gammaproteobacteria (19.7% of the winter library compared to 2.7% of the summer library) falling into 5 closely-related 0.03 distance bins that were affiliated with the GSO-EOSA-1 complex. Grzymski et al. (2012), in a metaproteomic survey of coastal waters at Palmer station, found GSO-EOSA-1 proteins composed a large fraction of all gammaproteobacterial proteins detected and were significantly more abundant in winter than summer (20% vs 3%). In a companion metaproteomic analysis of the same sites, Williams et al. (2012) confirmed this high abundance and seasonal pattern, although GSO-EOSA-1 appeared to be metabolically active at the surface in both summer and winter. Finally, Wilkins et al. (????) identified higher abundances of OTUs with high identity to sequenced GSO-EOSA-1 in the AZ than in the SAZ. 

Genomic and metaproteomic analyses of GSO-EOSA-1 representatives, particularly SUP05, have revealed the potential for carbon fixation via the Calvin cycle and sulfur oxidation, even in well-oxygenated waters (Walsh et al. 2009; Swan et al. 2011). Grzymski et al. (2012) estimated from rRNA abundances that between 18 and 37% of the winter bacterioplankton community comprises OTUs with the potential for chemolithoautotrophy, including GSO-EOSA-1, suggesting winter chemolithoautotrophy may contribute significantly to SO carbon fixation.

###Deltaproteobacteria

Deltaproteobacteria are rarely detected at abundance in global surface waters (see e.g. Venter et al. 2004), and this pattern appears to hold in the Southern Ocean (Murray and Grzymski 2007; West et al. 2008, Ghiglione and Murray 2011; Murray et al. 2010; Ducklow et al. 2011; Jamieson et al. 2012; Wilkins et al. ????). However, they may increase in abundance in mesopelagic waters (Wright et al. 1997; Pham et al. 2008; Zaballos et al. 2006). At a 3000 m deep site at the PF in the Drake Passage, López-García et al. (2001) detected several deltaproteobacterial 16S rDNA sequences, all of which clustered with the marine deltaproteobacterial clade SAR324 previously identified in the mesopelagic Sargasso Sea (Wright et al. 1997). Whole-genome analysis of SAR324 indicates an ecology that includes carbon fixation via the Calvin cycle and sulfur oxidation, as well as oxidation of methylated compounds (Swan et al., 2011). SAR324 may therefore be significant contributors to chemoautotrophy in the dark ocean (Swan et al., 2011).

###CFB

Bacteria of the group Cytophaga-Flavobacterium–Bacteroides (CFB) are cosmopolitan and abundant in the world ocean (Glöckner et al. 1999). While the CFB often form a major fraction of planktonic taxa (Fandino et al. 2001), they are particularly prevalent in particle-attached communities (DeLong et al. 1993) and are associated with blooming phytoplankton (Pinhassi 2004). Isolated CFB representatives have a well-described aptitude for the degradation of HMW DOM, particularly biopolymers which may be recalcitrant to utilisation by other bacterial heterotrophs (reviewed in Kirchman 2002), suggesting they play an important role in remineralization of primary production products. Of the CFB, the class Flavobacteria seem to be in the majority worldwide in both freshwater and marine environments (O'Sullivan et al. 2004; Cottrell et al. 2005) including the Southern Ocean (Abell and Bowman 2005a).

CFB in the SO are strongly biogeographically partitioned. Abell and Bowman (2005a), utilising DGGE and 16S sequencing with Flavobacteria-specific primers, found significantly higher abundance and diversity of particle-attached Flavobacteria in the nutrient- and phytoplankton-rich waters south of the PF relative to the HNLC waters of the Subantarctic. Likewise, a large-scale metagenomic analysis of the SO which identified the PF as a major biogeographic boundary found CFB contribute a large fraction of the variance between the zones, with higher abundance south of the PF (Wilkins et al. ????. This difference in abundance may be largely attributable to the low iron availability in the Subantarctic, which probably limits primary production (Boyd et al. 2007). Both natural and artificial iron fertilization events in the Subantarctic have resulted in high abundances of bacterial heterotrophs (Christaki et al. 2008; Oliver et al. 2004); West et al. (2008), identified the CFB as a major component of the bacterial response to blooms induced by natural iron input on the Kerguelen plateau. The higher abundance of CFB in the AZ may also relate to their prevalence in sea ice (Brinkmeyer et al. 2003; Brown and Bowman 2001), from which they would be released into AZ waters during seasonal melting. Two groups, the uncultured agg58 cluster and the genus Polaribacter, appear to dominate CFB populations and activity in the SO (Murray and Grzymski 2007; Abell and Bowman 2005a; Abell and Bowman 2005b; Obernosterer et al. 2011; West et al. 2008; Ghiglione and Murray 2011; Ducklow et al. 2011; Straza et al. 2010).

There is some evidence that planktonic and particle-attached CFB, rather than being an integrated population with cells opportunistically shifting between phases, may comprise at least partially distinct groups of phylotypes. In a mesocosm experiment examining colonisation of diatom detritus in SO seawater, 16S DGGE and sequencing analysis showed a large proportion of Flavobacterial phylotypes present in the planktonic phase failed to colonise detrital particles during the course of the experiment (Abell and Bowman 2005b). The authors suggest these phylotypes may be slower-growing, perhaps comprising a secondary group of colonisers which only come to dominate when the more accessible detrital nutrients have been exhausted and the primary colonisers have secreted useful secondary metabolites. Alternatively, some flavobacterial groups may not use particle attachment as a primary strategy. Questions around the relationship between particle-attached and free-living microbial communities emphasise the usefulness of size fractionation in molecular studies of marine microbial communities.

Kirchman (2002) suggests 16S clone libraries may systematically underestimate the abundance of CFB in environmental samples, noting that in two studies where both FISH and 16S analysis were employed at the same site there were common discrepancies in CFB abundance estimates between the two methods (Cottrell and Kirchman 2000; Eilers et al. 2000). Additionally, both FISH and PCR based methods may underestimate CFB abundance relative to metagenomic surveys, due to probe specificity biased against Bacteroidetes 16S rDNA (Cottrell et al. 2005; O'Sullivan et al. 2004).

Polaribacter is a gas-vacuolated, proteorhodopsin-expressing flavobacterial genus prevalent in Arctic and Antarctic seawater, and the genome indicates a preference for polymers obtained from algal detritus rather than labile exudates (e.g., taurine, polyamines) (Gonzalez et al., 2008). An Antarctic Peninsula coastal metagenome found Polaribacter-related sequences to be dominant in summer, consistent with an association with phytoplankton blooms and/or being seeded from melting sea-ice (Grzymski et al., 2012). Flavobacterial proteins (including those with the best matches to Polaribacter spp.) were similarly much more abundant in the summer versus winter metaproteome from the same sites, with components of TonB-dependent receptor systems predominating (Williams et al., 2012). 

###Cyanobacteria

Cyanobacteria, dominated by the genera Prochlorococcus and Synechococcus, are the most abundant photosynthetic organisms on Earth (Scanlan et al. 2009 and references therein), but little molecular research has been performed on their role in Southern Ocean ecosystems. This may be because it has been generally accepted that there are no cyanobacteria in AZ waters (Ghiglione and Murray 2011; Zubkov et al. 1998; Evans et al. 2011), although recent metagenomic (Wilkins et al. ????) and metaproteomic (Williams et al. 2012) results challenge this assumption. It is not infeasible that cyanobacteria survive at Antarctic temperatures, as (apparently psychrophilic or psychrotolerant) Synechococcus and Prochlorococcus strains have been identified in several marine-derived Antarctic lakes, including at sub-zero water temperatures (Bowman et al. 2000; Powell et al. 2005; Lauro et al. 2011). Regardless, it is clear that cyanobacteria, if present in the AZ, are at very low abundance and probably of little ecological significance. Cyanobacteria also appear to be at low abundance in the SAZ (Abell and Bowman 2005a; Wilkins et al. ????; Topping 2006).  


###Verrucomicrobia

The Verrucomicrobia is a recently described phylum that is ubiquitous in the marine environment, and appears to be composed of several physiologically distinct lineages (Freitas et al., 2012). A small number of representatives of Verrucomicrobia have been detected in the SO (Murray et al. 2010; West et al. 2008; Gentile et al. 2006; Murray and Grzymski 2007). Ghiglione and Murray (2011) reported a higher abundance of 16S rDNA clones affiliating with the Verrucomicrobia at a Kerguelen Island site relative to a site near Palmer Station on the Antarctic Peninsula. However, a metagenomic survey found higher levels of an OTU with best match to verrucomicrobium Coraliomargarita akajimensis in AZ waters than SAZ (Wilkins et al. ????). 


###Other bacteria

Bacteria of the phylum Planctomycetes have been detected at low abundance in SO molecular surveys (Gentile et al. 2006; López-García et al. 2001; Jamieson et al. 2012; Murray et al. 2010; Abell and Bowman 2005). Planctomycetes is emerging as a group of interest in marine microbial ecology, for example as performers of anaerobic ammonium oxidation (anammox) (Strous et al. 1999), such as indicated in the metaproteome from coastal West Antarctic waters (Williams et al., 2012). The latter study also detected Nitrospirae proteins pertaining to nitrite oxidation and carbon fixation via the reductive tricarboxylic acid cycle (Williams et al., 2012). Members of the Nitrospirae and Planctomyctes are therefore implicated in completing nitrification using nitrite generated by AOA and AOB in Antarctic waters.

Other bacterial groups have been reported at low abundance in Southern Ocean waters, including Actinobacteria (Bowman et al. 2003; Brinkmeyer et al. 2003; Abell and Bowman 2005; Gentile et al. 2006; Murray and Grzymski 2007; Murray et al. 2010; Ghiglione and Murray 2011; Bolhuis et al. 2011; Jamieson et al. 2012), Epsilonproteobacteria (Gentile et al. 2006; Murray and Grzymski 2007), and Firmicutes (Murray and Grzymski 2007; Guidice et al. 2011; Murray et al. 2010). Little is known about their respective ecological roles, although Actinobacteria have been associated with marine aggregates (Grossart et al. 2004); interestingly, their terrestrial counterparts have diverse HMW substrate degradation capabilities (reviewed in Kirchman 2008). A strong negative correlation has been reported between actinobacterial abundance and latitude in a global survey using 16S rDNA clone libraries (Pommier et al., 2007), with higher abundances in tropical and subtropical waters, as for Cyanobacteria.

##Archaea

DeLong et al. (1994) first reported the high abundance (up to 34%) of archaea in Antarctic coastal surface waters, a surprising discovery at a time when archaea were generally considered a rare group of strict extremophiles. The majority of rDNA clones they identified were affiliated with the Marine Group I Crenarchaeota (MGI), while the remainder represented the Group II Euryarchaeota. Subsequent rRNA-based studies are likewise in agreement that MGI are the dominant group of archaea in surface waters of coastal Antarctica, followed by Group II Euryarchaeota (Gerlache Strait, Massana et al., 1998; near Anvers Island, Murray et al. 1998). Further rRNA-based analysis showed the widespread distribution of Antarctic marine archaea both longitudinally as well north and south of the polar front (Murray et al. 1999; Topping et al. 2006; Jamieson et al. 2012), and the identification of a significant MGI community in benthic sediments on the Antarctic coast (Bowman and McCuaig 2003; Bowman et al. 2003).

For MGI, ammonium-oxidizing chemolithoautotrophy is likely the dominant metabolic lifestyle (Ingalls et al. 2006; Berg et al., 2007), suggesting they play major roles in nitrification and carbon fixation in the Southern Ocean. In a winter coastal Antarctic metaproteome, MGI proteins made up 30% of all identified proteins from bacteria or archaea; no MGI proteins were detected in the summer metaproteome (Williams et al., 2012). The winter metaproteome included MGI proteins pertaining to the 3-hydroxypropionate/4-hydroxybutyrate cycle, the pathway used by ammonia-oxidizing MGI for carbon fixation, and for ammonium transport and oxidation, supporting a nitrification role for Southern Ocean MGI (Williams et al., 2012). The complementary metagenomic analysis of Grzymski et al. (2012) proposed chemolithoautotrophy carried out by ammonia-oxidizing MGI and sulfur-oxidizing gammaproteobacteria (see GSO-EOSA-1, above) to be the major drivers of winter carbon fixation in AZ waters. In summer autotrophic carbon assimilation is driven by algal-driven oxygenic photoautotrophy, consistent with high light availability and intensity, whereas in the polar winter ‘dark’ chemoautotrophy by archaea and bacteria plays a major role in carbon fixation.  [maybe put this last sentence in a summary paragraph - TW]

Murray et al. (1998) found that total archaeal rRNA levels decreased during summer, and noted a negative correlation between archaeal rRNA levels and chlorophyll a concentration. Massana et al. (1998) also observed a decline in archaeal rRNA levels during spring. Church et al. (2003) found a significantly higher (44% increase) abundance of MGI in surface waters in winter compared to summer. Ammonia-oxidizing MGI have been shown to be especially sensitive to photoinhibition (Merbt et al., 2012), which might account for their decline during periods of extended illumination. It has also been speculated that the decline of archaea during spring/summer represents competition with non-archaeal microbes during phytoplankton blooming (Massana et al., 1998), or that the majority of MGI were chemoautotrophic and therefore more competitive compared to heterotrophs during carbon-scarce winter conditions (Murray et al., 1998). However, based on genomic evidence, Marine Group II Euryarchaeota are motile, proteorhodopsin-expressing photoheterotrophs that specialize in protein and lipid degradation (Iverson et al., 2012). This is consistent with results that this group, in contrast to MGI, was more relatively abundant at the surface than at depth (Massana et al., 1998). Murray et al. (1998) noted an increase in Group II Euryarchaeota in autumn in waters off Anvers Island; but otherwise the seasonal distribution of this group in Antarctic waters is not well understood.

The numerical dominance of MGI over other archaeal groups in surface and photic zone waters has also become well established (e.g. DeLong et al. 1998; Massana et al. 2000), although not in aphotic waters; López-García et al. (2001) detected only euryarchaeaotal sequences in a sample from 3000 m depth at the PF in the Drake Passage, including marine groups II and III and a novel marine group IV. However, these studies also illustrated a potential hazard of probe-dependant methods, namely the high variability of abundance estimates depending on probe design. For example, Simon et al. (1999) did not detect any DAPI-positive archaeal cells in a summer transect between the polar front and ice edge using the archaea-specific probes ARCH334 and ARCH915, in contrast to the results of several other studies reviewed herein. López-García et al. (2001) detected only one archaeal phylotype (an euryarchaeon) in their initial clone library constructed with one archaeal primer pair. This prompted the authors to design an additional five primer pairs, resulting in both a higher number and greater diversity of clones.

##Virioplankton

The 'viral shunt', by which nutrients are released via lysis from marine microorganisms and returned to the dissolved and particulate pools, may mediate the flux of a quarter of all organic matter in the microbial loop (Wilhelm and Suttle, 1999) and the viral release of iron from bacterioplankon may be cruical for phytoplanktonic growth (Poorvin et al. 2004). Viral production, and by inference the viral shunt, has been shown to be highly active in HNLC Subantarctic (Evans et al. 2009), iron-fertilized Subantarctic (Weinbauer et al. 2009) and coastal waters, where viral-mediated carbon flux may account for 50 - 100% of all heterotrophic production (Guixa-Boixereu 2002). Despite this crucial ecosystem role, however, molecular analysis of the diversity and function of SO virioplankton has been sparse. Two studies conducted by Short and Suttle (2002 and 2005) used probes with specificity to algal virus and cyanophage marker genes respectively, and succeeded in detecting both in SO waters. Williams et al. (2012) and Grzymski et al. (2012), in complementary metaproteomic and metagenomic studies of sites near Palmer station on the Antarctic Peninsula, also detected cyanophage (cyanobacterial virus) genes and proteins as well as a single major capsid protein from Phaeocystis pouchetii virus PpV01. Finally, Wilkins et al. (????) detected higher abundances of Ostreococcus virus in the AZ than the SAZ, and noted the presence of cyanophage in all SO waters. While these studies are only preliminary, they suggest that the more abundant viruses are phytoplanktonic predators. An extensive molecular survey of SO virioplankton (in the nature of e.g. Angly et al. 2006) would clearly be of great value.


ZZZ)))))


TODO end pasted block


\section{Oceanography of the Southern Ocean}


\subsection{Water masses and fronts}
\subsection{Effect of climate change}
\section{Role of the Polar Front in biogeography}
\section{Project questions and hypotheses}
