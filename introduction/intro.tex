\chapter{Introduction} 
\label{ch:intro}

\section{Some test text to preview layout \emph{etc.}}
The Roseobacter clade is an abundant and ecologically significant group of marine bacteria, found at high (> 15\%) abundance in most marine surface environments (\citet{Anonymous:2005hd} and references therein). Unlike some other major proteobacterial groups which are strongly associated with a particular ecological niche (e.g. the SAR11 clade), roseobacters have diverse metabolic abilities, with members capable (for example) of aerobic anoxygenic phototrophy \cite{Biebl:2005fp,Anonymous:mYN2Q-VF}, degradation of dimethylsulfoniopropionate (DMSP) by at least two pathways \cite{Anonymous:2007fs,Miller:2004jz}, carbon monoxide oxidation \cite{Anonymous:ltlffvSn} and heterotrophic utilisation of a broad range of substrates (reviewed in \cite{Anonymous:2008do}). Roseobacters are found in the planktonic fraction as well as in commensal association with phytoplankton and metazoans (reviewed in \citet{Anonymous:2005hd}).

\section{Microbial ecology of the Southern Ocean}

\section{Oceanography of the Southern Ocean}
\subsection{Water masses and fronts}
\subsection{Effect of climate change}

\section{Role of the Polar Front in biogeography}
\section{Project questions and hypotheses}
