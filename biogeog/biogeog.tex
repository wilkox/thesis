\chapter{Mesoscale biogeographic drivers of planktonic diversity}
\label{ch:biogeog}

\section{Introduction}

\section{Methods}

\subsection{Sampling}

\subsection{DNA extraction}

DNA extraction was performed using a modified version of the method described in \citet{Rusch:2007ez}.
Samples were thawed in a 37 \textdegrees{}C water bath.
Half of the storage buffer (\textapprox{} 10 mL) was decanted into a clean 50 mL centifuge tube.
If the volume decanted was less than 10 mL, the remainder was made with sterile water (Sigma-Aldrich, St.\ Louis, USA).
An equal volume of 50\% sucrose lysis buffer (50 mM TRIS-HCl, 40 mM ETDA, 0.75 M Sucrose, pH 8) was added such that the final concentration was 25\% sucrose lysis buffer.
A small spatula of lysozyme (TODO supplier) (final concentration \textapprox{} 2.5 mgmL$^{-1}$) and 1 mL TRIS-EDTA pH 8 (TODO recipe) was added.

The filter membrane was removed from the storage tube and cut in half aseptically.
One half was returned to the storage tube, which was refrozen at $-80$ \textdegrees{}C.
The remaining half was cut in half again, and one quarter-filter placed atop the other such that the biomass on each piece was facing outwards.
Keeping the filters together, they were cut into very fine (\textapprox{} 3 mm by 10 mm) strips, which were placed in the 50 mL centrifuge tube containing the buffer and lysozyme mixture.
This tube was mixed by gentle inverting, then tapped such that all filter strips collected at the bottom of the tube covered by lysis buffer.
The tube was then incubated in a 37 \textdegree{}C shaking water bath at 275 RPM for 30--60 min.

200 \microlitre{} of 20 mgmL$^{-1}$ Proteinase K (TODO supplier) was added to the tube, which was mixed by gentle inverting.
The tube was gently tapped such that all filter strips collected at the bottom covered by lysis buffer.
The tube was then subjected to three freeze--thaw cycles, each cycle consisting of 20--30 minutes in a $-80$ \textdegrees{}C freezer followed by 20--30 minutes in a 55 \textdegrees{}C water bath.
After the final complete thaw, 200 \microlitre{} of 20 mgmL$^{-1}$ Proteinase K (TODO supplier) and 2 mL of 10\% SDS (TODO supplier) were added to the tube.
The tube was mixed by gentle inverting then gently tapped such that all filter strips collected at the bottom covered by lysis buffer.
It was then incubated in a 55 \textdegrees{}C shaking water bath at 175 RPM for two hours.

The supernatent was pipetted from the tube and split evenly into two new 50 mL centrifuge tubes.
An equal volume of buffer-saturated phenol (TODO supplier) was added to each tube, which were mixed by gentle inversion.
The mixtures were then fractionated in a fixed-angle rotor centrifuge for 15 min at 3700 RPM at room temperature.
The bottom layer of each tube was removed by pipette into a new 50 mL centrifuge tube.
Each of these two tubes was then made to 50 mL with sterile water (TODO supplier).
After mixing by gentle inverting, each 50 mL mixture was then split evenly into two new 50 mL centrifuge tubes, resulting in four tubes each containing 25 mL of mixture.
These tubes were then made to 50 mL with 1-propanol (TODO supplier).
The mixtures were homogenised by gentle inversion and incubated at 4 \textdegrees{}C overnight.

Following incubation, the tubes were centrifuged using a fixed-angle rotor for 30 min at 7500 RPM and room temperature.
The supernatent was removed by decanting, and the tubes left to sit until the remaining supernatent (\textapprox{} 1 mL) collected at the bottom over the precipitated pellet.
The pellet was then resuspened by gentle pipetting with a genomic tip, and the suspension placed in a new 1.5 mL microcentrifuge tube (four tubes total).
These tubes were then centrifuged in a microcentrifuge for 10 minutes at 13,000 RPM and room temperature.
The supernatent was removed by pipette and the tubes placed in a 37 \textdegrees{}C heat block with the lids opened and covered by a sterile Kimwipe (TODO supplier) for 10 min, or longer if the supernatent did not evapourate completely in that time.
93.75 \microlitre{} TRIS-EDTA pH 8 (TODO recipe) was added to each tube, and the tubes were incubatd at 4 \textdegrees{}C for one hour to allow the DNA pellet to redissolve.

After this incubation, the pellets were gently pipetted with a genomic tip to ensure complete resuspension.
The suspensions from all four tubes were combined, and an additional 750 \microlitre{} TRIS-EDTA (TODO recipe) added.
This was then split evenly into two new 1.5 mL microcentrifuge tubes (\textapprox{} 562.5 \microlitre{} per tube).

750 \microlitre{} of buffer-saturated phenol (TODO supplier) was added to each tube, and the tubes mixed gently by inversion until a visible emulsion formed.
Phase seperation was performed by centrifugation for 5 min at 13,000 RPM and room temperature.
The upper (aqueous) phase was removed to a new 1.5 mL microcentrifuge tube using a genomic tip.

750 \microlitre{} of phenol-chloroform-isoamyl alcohol mixture (TODO supplier and ratio) was added to each tube, and the tubes mixed by gentle inversion until a visible emulsion formed.
Phase seperation was performed by centrifugation for 5 min at 13,000 RPM and room temperature.
The upper (aqueous) phase was removed to a new 1.5 mL microcentrifuge tube using a genomic tip.

75 \microlitre{} of 3 M sodium acetate (pH 8) and 750 \microlitre{} 1-propanol (TODO supplier) was added to each tube.
The tubes were centrifuged at 13,000 RPM and room temperature for 30 min to precipitate the DNA.
The supernatent was removed by pipetting, and 100 \microlitre{} of 70\% ethanol added.
The tubes were centrifuged again at 13,000 RPM and room temperature for 5 min.
The supernatent was removed by pipetting and the DNA pellet dried in a 37 \textdegrees{}C heat block.
The DNA was dissolved overnight in 40--200 \microlitre{} TRIS-EDTA (TODO recipe), depending on the expected yield.

\section{Results}

\section{Discussion}
