\begin{figure}[!ht]
  \centering
  \includegraphics[width=\textwidth]{../advection/dbRDA.png}
  \caption[dbRDA ordination of relationship between environment and community.]{dbRDA ordination of the distLM model describing the relationship between the BEST-selected set of predictor physicochemical variables (pressure, oxygen, temperature, salinity, silicate, and nitrate) and the taxonomic dissimilarity between samples. Vectors represent the effect of each predictor variable on the two visualised axes. Vector length corresponds to the relative size of the effect, while direction represents the correlations to the two displayed axes. The first axis (dbRDA1) captures 64\% of fitted and 37\% of total variation between the samples' taxonomic profiles; the second (dbRDA2) captures 14\% of fitted and 8\% of total variation. Antarctic Intermediate Waters (AAIW), light blue stars; Subantarctic Mode Water (SAMW), orange crosses; Antarctic Bottom Water (AABW), dark blue squares; Antarctic Zone (AZ), green circles; Polar Frontal Zone (PFZ), yellow triangles; Circumpolar Deep Water (CDW), red diamonds.}
  \label{fig:dbRDA}
\end{figure}
