\chapter[The advection effect]{The advection effect as a driver of microbial biogeography}
\label{ch:advection}

\previouslypublished{Sections of this chapter have been previously published in}

\section{Summary}

\section{Introduction}

The central goal of microbial biogeography is to understand how the distribution and abundance of microorganisms are shaped by their physical context.
The Baas Becking hypothesis --- that ``\textit{everything is everywhere}, but, \textit{the environment selects} \cite{Becking:1934um, deWit:2006de}'' --- posits that the rapid dispersal of microorganisms means microbial community structure is determined entirely by environmental selection.
This stands in contrast to macroorganism biogeography, which has long been recognised as being under the control of historical (in addition to contemporary environmental) factors, particularly spatial influences such as barriers to dispersal.
Microbial biogeography studies have begun to show that historical factors may also shape the distribution of microorganisms \cite{Martiny:2006jy}, e.g.\ a correlation between spatial and genetic distance (a ``distance effect'') in fluorescent \genus{Pseudomonas} strains in soils \cite{Cho:2000tn}.
This study, among others \cite{Ramette:2007bb,Storch:2008tq}, also demonstrated the importance of taxonomic resolution in describing such biogeographic patterns.
Other studies have found that dispersal potential varies between microbial species, leading to different or absent distance effects \cite{Bissett:2010wj}.
When combined with contemporary environmental selection (``environment effect''), distance effects explain some but not all variation between microbial communities, and the mechanism(s) by which a distance effect arises are not always clear \cite{Hanson:2012cb}.

In the ocean, several recent studies have found that microbial communities can be endemic to hydrographically distinct water masses.
Surveys in the Arctic \cite{Galand:2009hy} and North Atlantic \cite{Agogue:2011fm} oceans have found that bacterial assemblages within the same water mass can be similar across a range of thousands of kilometres, but assemblages can differ between water masses across a range of hundreds of meters.
Water masses are defined by their distinct physicochemical properties, so such patterns do not directly imply the existence of factors beyond environmental selection.
However, in some cases a water mass-community relationship has been shown to persist even when environment effects are statistically controlled for \cite{Hamilton:2008tp, Hamdan:2013ko}.

During the study on the biogeographic effect of the \ac{PF} described earlier in this volume, it was found that microbial communities in surface waters of the Mertz Glacier region, a site of deep water formation, were very similar to those at the bottom of the water column, despite the very different environmental conditions (see \secreft{ch:deepappendix}).
One hypothesis explaining both this observation and water mass endemicity is that microbial assemblages are influenced by the advection (physical transport) of cells by ocean currents. 
Higher dispersal rates cause the microbial community composition at a given site to increasingly resemble the dispersed colonisers, and less reflect local environmental selection and stochastic effects such as genetic drift \cite{Hanson:2012cb}.
Hence, it would be expected that locations that are closely connected by advection (e.g.\ those within the same water mass, or different levels of the water column at a site of deep water formation) would have more similar compositions than those that are not, even when the environment effect is accounted for.
Indeed, advection is often invoked to explain observations of microbial diversity or abundance which do not seem attributable to environmental selection \citep[e.g.][]{Sul:2013in, Ghiglione:2012ei, Giebel:2009hr, Lauro:2007bf}.
The exchange of very small volumes of water between marine microbial mesocosms has been found to greatly reduce their \textbeta-diversity even under consistent environmental conditions \cite{Declerck:2013cz}.
This suggests that advection of even small numbers of cells could have a large homogenizing effect independent of environmental selection.
However, the existence of a relationship between advection and community composition that is independent of environment and distance effects has not been directly tested.

The \ac{SO} is composed of several water masses, which are physicochemically distinct but linked by circulation (see \secreft{ch:intro} for a full description; see also \autoref{fig:advectionsamplemap}).
This study aimed to determine whether advection shapes the community structure of bacteria and archaea, independent of environment and distance effects.
By sampling each of the \ac{SO} water masses (depths from surface to \textapprox{}6 km), dissimilarity between microbial communities over a large spatial distance (\textapprox{}3000 km) and range of environments could be determined, in order to test whether advection played a role in shaping their composition.

\section{Methods}

\subsection{Sampling}

\begin{figure}
  \centering
  \includegraphics[width=\textwidth]{../advection/advectionsamplemap.png}
  \caption[Map showing sites of samples used in the advection study]{Antarctic Intermediate Waters (AAIW), light blue stars; Subantarctic Mode Water (SAMW), orange crosses; Antarctic Bottom Water (AABW), dark blue squares; Antarctic Zone (AZ), green circles; Polar Frontal Zone (PFZ), yellow triangles; Circumpolar Deep Water (CDW), red diamonds; sea surface, blue dashed horizontal line. Bathymetry is an approximate representation for 115\textdegree{} E, and is indicative only.}
  \label{fig:advectionsamplemap}
\end{figure}

Sampling\footnote{Sampling was performed by David Wilkins, Timothy J.\ Williams and Sheree Yau.} was conducted on board the RSV \textit{Aurora Australis} during cruise V3 from January 25th--February 12th 2012.
This cruise occupied two latitudinal transects: one from Hobart, Australia (\textapprox{}44\textdegree{} S) to the Mertz Glacier, Antarctica (\textapprox{}67\textdegree{} S), within a longitudinal range of 140--150\textdegree{} E; the second from waters north of Cape Poinsett, Antarctica (\textapprox{}66\textdegree{} S) to Freemantle, Australia (\textapprox{}32\textdegree{} S) within a longitudinal range of 110--120\textdegree{} E.

At each station, \textapprox{}250--560 L of seawater was pumped from \textapprox{}1.5--2.5 m below the sea surface into drums stored at ambient temperature on deck.
At some stations, an additional sample was taken from the \ac{DCM}, as determined by chlorophyll fluorescence measurements taken from a \ac{CTD} (SeaBird, Bellevue, USA) cast at each sampling station.
In the case of deep and intermediate water samples, \textapprox{}120--240 L of seawater was collected from Niskin bottles attached to a \ac{CTD}.
The depths were selected based on temperature, salinity and dissolved oxygen profiles established by \ac{CTD} casts at each sampling station to capture water from the targeted water mass.
Profiles were generated on the \ac{CTD} downcast, and bottle firings (i.e.\ sample collection) on the returning upcast at the selected depths.

Seawater samples were prefiltered through a 20 \micron{} plankton net, then filtrate was captured on sequential 3.0 \micron{} 0.8 \micron{} and 0.1 \micron{} 293 mm polyethersulfone membrane filters (Pall, Port Washington, USA), and immediately stored at $-20$ $^\circ$C \cite{Rusch:2007ez,Ng:2010cd}.

\subsection{DNA extraction}

DNA extraction was performed using a modified version of the phenol-chloroform method described in \citet{Rusch:2007ez}.
Samples were thawed in a 37 \textdegree{}C water bath.
Half of the storage buffer (\textapprox{}10 mL) was decanted into a clean 50 mL centrifuge tube.
If the volume decanted was less than 10 mL, the difference was made with sterile water (Sigma-Aldrich, St.\ Louis, USA).
An equal volume of 50\% sucrose lysis buffer (50 mM TRIS-HCl, 40 mM EDTA, 0.75 M Sucrose, pH 8) was added such that the final concentration was 25\% sucrose lysis buffer.
A small pinch of lysozyme (Sigma-Aldrich, St.\ Louis, USA) (final concentration \textapprox{}2.5 mg/mL) and 1 mL TRIS-EDTA (10 mM TRIS, 1 mM EDTA, pH 8) was added.

The filter membrane was removed from the storage tube and cut in half aseptically.
One half was returned to the storage tube, which was refrozen at $-80$ \textdegree{}C.
The remaining half was cut in half again, and one quarter-filter placed atop the other such that the biomass (filtrand) on each piece was facing outwards.
Keeping the filters together, they were cut into very fine (\textapprox{}3 mm by 10 mm) strips, which were placed in the 50 mL centrifuge tube containing the buffer and lysozyme mixture.
This tube was mixed by gentle inversion, then tapped such that all filter strips collected at the bottom of the tube and were covered by lysis buffer.
The tube was then incubated in a 37 \textdegree{}C shaking water bath at 275 RPM for 30--60 min.

200 \microlitre{} of 20 mg/mL Proteinase K (Sigma-Aldrich, St.\ Louis, USA) was added to the tube, which was mixed by gentle inversion.
The tube was gently tapped such that all filter strips collected at the bottom covered by lysis buffer.
The tube was then subjected to three freeze-thaw cycles, each cycle consisting of 20--30 min in a $-80$ \textdegree{}C freezer followed by 20--30 min in a 55 \textdegree{}C water bath.
After the final complete thaw, 200 \microlitre{} of 20 mg/mL Proteinase K and 2 mL of 10\% SDS (Sigma-Aldrich, St.\ Louis, USA) were added to the tube.
The tube was mixed by gentle inversion then gently tapped such that all filter strips collected at the bottom covered by lysis buffer.
It was then incubated in a 55 \textdegree{}C shaking water bath at 175 RPM for two hours.

The supernatant was pipetted from the tube using a genomic tip and split evenly into two new 50 mL centrifuge tubes.
An equal volume of buffer-saturated (10 mM TRIS HCl, 1 mM EDTA, pH 8) phenol (Sigma-Aldrich, St.\ Louis, USA) was added to each of the tubes, which were mixed by gentle inversion.
The mixtures were then fractionated in a fixed-angle rotor centrifuge for 15 min at 3700 RPM at room temperature.
The bottom layer of each tube was removed by pipette into a new 50 mL centrifuge tube.
Each of these two tubes was then made to 50 mL with sterile water (Sigma-Aldrich, St.\ Louis, USA).
After mixing by gentle inversion, each 50 mL mixture was then split evenly into two new 50 mL centrifuge tubes, resulting in four tubes each containing 25 mL of mixture.
These tubes were then made to 50 mL with 1-propanol (Sigma-Aldrich, St.\ Louis, USA).
The mixtures were homogenised by gentle inversion and incubated at 4 \textdegree{}C overnight.

Following incubation, the tubes were centrifuged using a fixed-angle rotor for 30 min at 7500 RPM and room temperature.
The majority of the supernatant was removed by decanting, and the tubes left to sit until the remaining supernatant (\textapprox{}1 mL) collected at the bottom over the precipitated pellet.
The pellet was then resuspended by gentle pipetting with a genomic tip, and the suspension placed in a new 1.5 mL microcentrifuge tube (four tubes total).
These tubes were then centrifuged in a microcentrifuge for 10 minutes at 13,000 RPM and room temperature.
The supernatant was removed by pipette and the tubes placed in a 37 \textdegree{}C heat block with the lids opened and covered by a sterile KimWipe (Kimberly-Clark, Irving, USA) for 10 min, or longer if the supernatant did not evaporate completely in that time.
93.75 \microlitre{} of TRIS-EDTA was added to each tube, and the tubes were incubated at 4 \textdegree{}C for one hour to allow the DNA pellet to redissolve.

After this incubation, the pellets were gently pipetted with a genomic tip to ensure complete resuspension.
The suspensions from all four tubes were combined, and an additional 750 \microlitre{} of TRIS-EDTA added.
This was then split evenly into two new 1.5 mL microcentrifuge tubes (\textapprox{}562.5 \microlitre{} per tube).

750 \microlitre{} of buffer-saturated phenol was added to each tube, and the tubes mixed gently by inversion until a visible emulsion formed.
Phase separation was performed by centrifugation for 5 min at 13,000 RPM and room temperature.
The upper (aqueous) phase was removed to a new 1.5 mL microcentrifuge tube using a genomic tip.

750 \microlitre{} of phenol-chloroform-isoamyl alcohol (25:24:1) mixture (Sigma-Aldrich, St.\ Louis, USA) was added to each tube, and the tubes mixed by gentle inversion until a visible emulsion formed.
Phase separation was performed by centrifugation for 5 min at 13,000 RPM and room temperature.
The upper (aqueous) phase was removed to a new 1.5 mL microcentrifuge tube using a genomic tip.

75 \microlitre{} of 3 M sodium acetate (pH 8) and 750 \microlitre{} of 1-propanol was added to each tube.
The tubes were centrifuged at 13,000 RPM and room temperature for 30 min to precipitate the DNA.
The supernatant was removed by pipetting, and 100 \microlitre{} of 70\% ethanol added.
The tubes were centrifuged again at 13,000 RPM and room temperature for 5 min.
The supernatant was removed by pipetting and the DNA pellet dried in a 37 \textdegree{}C heat block.
The DNA was dissolved overnight in 40--200 \microlitre{} of TRIS-EDTA, depending on the expected yield.

\section{Results}

\section{Discussion}
